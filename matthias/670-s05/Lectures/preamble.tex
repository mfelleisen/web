\usepackage{a210}
\usepackage{palatino}
\usepackage{slatex}
\usepackage{epsf}
\usepackage{mztp}
\usepackage{html}

% ---------------------------------------------------------------
% special slatex symbols

\setspecialsymbol{fst}{{\it first\/}} % otherwise we get bad first

\setspecialsymbol{newline}{\relax}
\setspecialsymbol{anewline}{\relax}

\setspecialsymbol{<-ir}{$<_{ir}$}

\setspecialsymbol{Celsius->Fahrenheit}{{\arrowname{\it Celsius\/}{\it Fahrenheit\/}}}
\setspecialsymbol{Fahrenheit->Celsius}{{\arrowname{\it Fahrenheit\/}{\it Celsius\/}}}
\setspecialsymbol{dollar->euro}{{\arrowname{\it dollar\/}{\it euro\/}}}

\setspecialsymbol{inches->cm}{\arrowname{{\it inches\/}}{{\it cm\/}}}
\setspecialsymbol{feet->inches}{\arrowname{{\it feet\/}}{{\it inches\/}}}
\setspecialsymbol{yards->feet}{\arrowname{{\it yards\/}}{{\it feet\/}}}
\setspecialsymbol{rods->yards}{\arrowname{{\it rods\/}}{{\it yards\/}}}
\setspecialsymbol{furlongs->rods}{\arrowname{{\it furlongs\/}}{{\it rods\/}}}
\setspecialsymbol{miles->furlongs}{\arrowname{{\it miles\/}}{{\it furlongs\/}}}

\setspecialsymbol{feet->cm}{\arrowname{{\it feet\/}}{{\it cm\/}}}
\setspecialsymbol{yards->cm}{\arrowname{{\it yards\/}}{{\it cm\/}}}
\setspecialsymbol{rods->inches}{\arrowname{{\it rods\/}}{{\it inches\/}}}
\setspecialsymbol{miles->feet}{\arrowname{{\it miles\/}}{{\it feet\/}}}

\setspecialsymbol{time->seconds}{\arrowname{\it time\/}{\it seconds\/}}

\setspecialsymbol{hours->wages}{\arrowname{\it hours\/}{\it wages\/}} 

\setspecialsymbol{boxed:rel-op}{\fbox{{\it rel-op\/}}}
\setspecialsymbol{boxed:op1}{\fbox{{\it op1\/}}}
\setspecialsymbol{boxed:op2}{\fbox{{\it op2\/}}}
\setspecialsymbol{boxed:doll}{\fbox{{\sf 'doll}}}
\setspecialsymbol{boxed:car}{\fbox{{\sf 'car}}}
\setspecialsymbol{boxed:s}{\fbox{$s$}}
\setspecialsymbol{boxed:lt}{\fbox{$<$}}
\setspecialsymbol{boxed:gt}{\fbox{$>$}}
\setspecialsymbol{boxedRO}{\fbox{{\it rel-op}}}
\setspecialsymbol{boxed:ltir}{\fbox{$<_{ir}$}}
\setspecialsymbol{<-ir}{$<_{ir}$}
\setspecialsymbol{C->F}{\arrowname{\it C\/}{\it F\/}}
\setspecialsymbol{boxed:C->F}{\fbox{{\arrowname{\it C\/}{\it F\/}}}}
\setspecialsymbol{boxed:IR-name}{\fbox{{\it IR-name}}}
\setspecialsymbol{boxed:f}{\fbox{{\it f}}}

\setspecialsymbol{number->string}{{\arrowname{{\it number\/}}{{\it string\/}}}}
\setspecialsymbol{string->symbol}{\arrowname{{\it string\/}}{{\it symbol\/}}}
\setspecialsymbol{string->number}{\arrowname{{\it string\/}}{{\it number\/}}}
\setspecialsymbol{pad->gui}{\arrowname{{\it pad\/}}{{\it gui\/}}}

\setspecialsymbol{hours->wages}{\arrowname{\it hours\/}{\it wages\/}} 
\setspecialsymbol{hours->wages2}{\arrowname{\it hours\/}{\it wages2\/}} 
\setspecialsymbol{hours->wages3}{\arrowname{\it hours\/}{\it wages3\/}} 

\setspecialsymbol{test-hours->wages}{\arrowname{\it test-hours\/}{\it wages\/}} 

\setspecialsymbol{file->list-of-lines}{\arrowname{{\it file\/}}{{\it list-of-lines}}}
\setspecialsymbol{file->list-of-checks}{\arrowname{{\it file\/}}{{\it list-of-checks}}}


\setconstant{
make-posn posn-x posn-y set-posn-x! set-posn-y! posn?
make-aaa aaa-xx aaa-yy set-aaa-xx! set-aaa-yy! aaa?
make-bbb bbb-zz bbb-ww set-bbb-zz! set-bbb-ww! bbb?
make-node node-name node-to set-node-name! set-node-to! node?
  node-visited set-node-visited!
make-person person-name person-social person-father person-mother person-children
  set-person-father! set-person-mother! set-person-children!
  person?
make-entry entry-name entry-number set-entry-name! set-entry-number! entry? 
make-cheerleader
make-circle circle-center circle-radius circle? 
make-square square-nw square-length square? 
make-hand hand-rank hand-suit hand-next hand? set-hand-next! 
}


\setconstant{
make-mail
make-file file-name file-size file-content file?
make-dir dir-name dir-dirs dir-files file-content dir?
make-node node-ssn node-name node-left node-right node?
make-child child-father child-mother child-name child-date child-eyes child?
make-posn posn-x posn-y posn?
make-add add-left add-right add? 
make-mul mul-left mul-right mul? 
make-prim prim-name prim-left prim-right prim? 
make-parent parent-children parent-name parent-eyes parent-date parent?
make-roman roman-grp roman?
make-italics italics-grp italics?
make-bold-face bold-face-grp bold-face?
make-wp wp-header wp-body wp?
make-phone-record phone-record-name phone-record-number phone-record?
}


\setconstant{
make-vec vec-x vec-y vec?
make-posn posn-x posn-y posn?
make-point point-x point-y point-z point? 
make-movie movie-title movie-producer movie? 
make-star star-name star-first star-instrument star-sales star? 
make-student student-last student-first student-teacher
make-airplane airplane? 
make-circle circle-center circle-radius circle? 
make-square square-nw square-length square? 
make-word make-time
make-entry entry-name entry-zip entry-phone
make-ball ball-posn
}

\setconstant{
make-check-box
make-radio-box
make-vec vec-x vec-y vec?
make-posn posn-x posn-y posn?
make-point point-x point-y point-z point? 
make-movie movie-title movie-producer movie? 
make-star star-last star-first star-instrument star-sales star? 
make-student student-last student-first student-teacher
make-airplane airplane? 
make-circle circle-center circle-radius circle? 
make-square square-nw square-length square? 
make-word make-time
make-entry entry-name entry-zip entry-phone
make-ball ball-posn
make-pair pair-left pair-right pair? 
make-ir ir-name ir-price ir?
make-er er-name er-rate er?
}


\setconstant{
make-posn posn-x posn-y posn? 
make-ball ball-x ball-y ball-delta-x ball-delta-y ball? 
make-rr
}


\setconstant{
make-hide hide-it hide?
make-loc loc-con loc?
}

% ---------------------------------------------------------------
% for dependence.tex 
\newcounter{foops}


% ---------------------------------------------------------------
% for prologue and epilogue

%% #1: quote
%% #2: title 
%% #3: toc line 
\def\ppart#1{\noindent {\Large #1}\\
  \noindent \rule{\textwidth}{4pt}\\
  \vskip 20ex}

\def\psect#1{\subsection*{#1}}

\htmlonly
\def\ppart#1{\part{#1}}
\def\psect#1{\subsection{#1}}
\endhtmlonly

% ---------------------------------------------------------------
% html coversion of little things
\def\ul#1{\underline{#1}}

\htmlonly
\def\ul#1{%
 \begin{rawhtml}<u>\end{rawhtml}#1\begin{rawhtml}</u>\end{rawhtml}%
}
\endhtmlonly

\def\anote#1{\medskip\noindent{\bf #1}:\ }

% ---------------------------------------------------------------
%% engl

\def\hints{\noindent{\bf Hints:} }
\def\hint{\noindent{\bf Hint:} }

% ---------------------------------------------------------------
% FIGURES
\def\beginfig{\begin{figure*}[htbp]{\noindent\hrulefill\par}\small}
\def\endfig{{\noindent\hrulefill\par}\end{figure*}}

\renewcommand{\topfraction}{1}
\renewcommand{\dbltopfraction}{1}
\renewcommand{\bottomfraction}{1}
\renewcommand{\textfraction}{0}

\renewcommand{\floatpagefraction}{.90} % for one-column
\renewcommand{\dblfloatpagefraction}{.99} % for two-.

% ---------------------------------------------------------------
% LABELS

\pagestyle{headings}

\def\intermezzo#1#2{%
 \refstepcounter{section}
 \setcounter{subsection}{0} 
 \section*{#1:~#2}
 \addcontentsline{toc}{part}{#1:~#2}
 \markboth{{\footnotesize #1}}{{\footnotesize #2}}
 }
\def\subintermezzo#1{%
 \refstepcounter{subsection}\subsection*{#1} 
 \addcontentsline{toc}{subsection}{#1}
 \setcounter{exercise}{0}}

\htmlonly
\def\intermezzo#1#2{%
 \part{#1~#2}
 \section{#1~#2}
 }
\def\subintermezzo#1{%
 \subsection{#1}}
\endhtmlonly

% ---------------------------------------------------------------
% NOTES, CROSS REFERENCES

% #1: text for marginal para
\def\mp#1{\marginpar{\footnotesize \raggedright #1}}

\def\plt#1{\marginpar{\footnotesize \raggedright {\sc PLT}: #1}}

\def\teacher{\epsfbox{icons/teacher1.ps}}
\def\drscheme{\epsfxsize=.5in\epsfbox{icons/plt.ps}}

\def\drstuff#1{{\footnotesize
{\sf
\begin{flushright}
#1
\end{flushright}}}}

\def\drhint#1#2{% 
 \marginpar{\hfill\drscheme}
 \marginpar{\drstuff{#2}}}

\long\def\note#1{\relax}    %%{\marginpar[\hfill\teacher]{\teacher}}

\def\hand{\epsfbox{icons/hand.right.ps}}
\def\solution#1{\relax}

\htmlonly 
\def\plt#1{\relax}
\def\teacher{\htmladdimg{../icons/teacher1.gif}}
\def\drscheme{\htmladdimg{../icons/plt.gif}}
\def\drhint#1#2{\htmlref{\drscheme}{#1}}

\long\def\note#1{\\ \teacher \\ }  %% Don't forget the space!!!

\def\hand{
\begin{rawhtml}
<img src="../icons/hand.right.gif">
\end{rawhtml}
}

\def\solution#1{\htmladdnormallink{Solution}{../Solutions/#1.html}}
% {\hand}
\endhtmlonly

% ---------------------------------------------------------------
\def\achild#1#2#3{\begin{picture}(75,30)
\put(00,00){\line(+1,+0){70}}
\put(70,00){\line(+0,+1){30}}
\put(70,30){\line(-1,+0){70}}
\put(00,30){\line(+0,-1){30}}
\put(00,19){\mbox{~#1~~(#2)}}
\put(00,05){\mbox{~Eyes:~~ #3}}
\end{picture}}

% ---------------------------------------------------------------
% PS FILES FOR CROSS REFERENCES
\def\spades{\epsfbox{icons/spades.ps}}
\def\clubs{\epsfbox{icons/clubs.ps}}

\def\planets{\centerline{\epsfxsize=5in\epsfbox{icons/planets0.ps}}}
\def\cards{\centerline{\epsfxsize=5in\epsfbox{icons/cards.ps}}}
\def\ricetour{\centerline{\epsfxsize=5in\epsfbox{icons/rice-tour.ps}}}
\def\morerice{\centerline{\epsfxsize=5in\epsfbox{icons/rice-tour2.ps}}}
\def\intersection{\centerline{\epsfxsize=1in\epsfbox{icons/traffic.ps}}}
\def\multiTL{\centerline{\epsfysize=1in\epsfbox{icons/multiple-traffic.ps}}}

\def\drawfigpic{
 \begin{center}
 \hbox to \textwidth{
  \hfill
  \epsfxsize=1.5in \epsfbox{icons/hang0.ps} 
  \hfill
  \epsfxsize=1.5in \epsfbox{icons/hang1.ps}
  \hfill
  \epsfxsize=1.5in \epsfbox{icons/hang2.ps}
  \hfill}
  \end{center}}
\def\lookuppic{
 \begin{center}
 \hbox to \textwidth{
  \hfill
  \epsfxsize=1.5in \epsfbox{icons/lookup-gui0.ps} 
  \hfill
  \epsfxsize=1.5in \epsfbox{icons/lookup-gui1.ps}
  \hfill
  \epsfxsize=1.5in \epsfbox{icons/lookup-gui2.ps}
  \hfill}
  \end{center}}
\def\pbpic{\smallskip
 \centerline{\epsfxsize=2.5in \epsfbox{icons/phonebook.ps}}
}

\def\hmguipic{\smallskip
 \centerline{\epsfxsize=2.5in \epsfbox{icons/hang-gui.ps}}
}

\def\riotpic{\smallskip
 \centerline{\epsfxsize=1.in \epsfbox{icons/riot.ps}}
}

\def\guipad{
 \begin{center}
 \hbox to \textwidth{
  \hfill
  \epsfysize=1.5in 
  \epsfbox{icons/vp.ps} 
  \hspace{2in}
  \epsfysize=1.5in 
  \epsfbox{icons/cal.ps}
  \hfill}
  \end{center}}

\def\toytable{
 \begin{center}
 \begin{tabular}{l|r|c}\hline
  \hbox to 1.2in{Symbol} & \hbox to 1.2in{Price} & \hbox to 1.2in{Image} \\ \hline 
  && \\
  robot  & 11.95 & \epsfbox{icons/robot.ps} \\
  && \\
   doll  & 19.95 & \epsfbox{icons/doll.ps} \\
  && \\
 rocket  & 29.95 & \epsfbox{icons/rocket.ps} \\
  \vdots & \vdots & \vdots \\
  \end{tabular}
 \end{center}}

\def\bezierpic{
 \begin{center}
 \hbox to \textwidth{
  \hfill
  \epsfxsize=1.5in \epsfbox{icons/bezier1.ps} 
  \hfill
  \epsfxsize=1.5in \epsfbox{icons/bezier2.ps}
  \hfill
  \epsfxsize=1.5in \epsfbox{icons/bezier3.ps}
  \hfill}
  \end{center}}
\def\taskqpic{
 \begin{center}
 \hbox to \textwidth{
  \hfill
  \epsfxsize=1.5in \epsfbox{icons/taskq1.ps} 
  \hfill
  \epsfxsize=1.5in \epsfbox{icons/taskq2.ps}
  \hfill
  \epsfxsize=1.5in \epsfbox{icons/taskq3.ps}
  \hfill}
  \end{center}}
\def\movingpic{ %% DONT COPY!!!!!!!!!
  \smallskip
  \centerline{\epsfxsize=\textwidth\epsfbox{icons/moving1.ps}}
  \smallskip
  \centerline{\epsfxsize=\textwidth\epsfbox{icons/moving2.ps}}}
\def\treepic{
 \begin{center}
 \hbox to \textwidth{
  \hfill
  {\epsfxsize=1.5in\epsfbox{icons/tree1.ps}}
  \hfill
  {\epsfxsize=1.5in\epsfbox{icons/tree2.ps}}
  \hfill}
 \end{center}}
\def\siepic{
 \begin{center}
 \hbox to \textwidth{  \hfill
  \epsfxsize=1.5in \epsfbox{icons/sie1.ps} 
  \hfill
  \epsfxsize=1.5in \epsfbox{icons/sie2.ps} 
  \hfill
  \epsfxsize=1.5in \epsfbox{icons/sie3.ps}
  \hfill}
  \end{center}}
\def\trafficpic{
 \begin{center}
 \hbox to \textwidth{  \hfill
  \epsfysize=.5in 
  \epsfbox{icons/next.ps} 
  \hfill
  \epsfysize=1in 
  \epsfbox{icons/green.ps} 
  \hfill
  \epsfysize=1in 
  \epsfbox{icons/yellow.ps} 
  \hfill
  \epsfysize=1in 
  \epsfbox{icons/red.ps}
  \hfill}
  \end{center}}

\def\siepicB{
  \begin{center}
  \hbox to \textwidth{\hfill \epsfxsize=1.5in \epsfbox{icons/sie2.ps} \hfill}
  \end{center}}
\def\diffpic{\centerline{\epsfxsize=4in \epsfbox{icons/differentiate.eps}}}
\def\integratepic{\centerline{\epsfxsize=4in \epsfbox{icons/integrate.eps}}}
\def\machinepic{\centerline{\epsfxsize=4in \epsfbox{icons/computer-pic.ps}}}

\htmlonly 
\def\spades{
  \begin{rawhtml}
  <img src=../icons/spades.gif alt="[spades]">
  \end{rawhtml}}
\def\clubs{
  \begin{rawhtml}
  <img src=../icons/clubs.gif alt="[clubs]">
  \end{rawhtml}}

\def\planets{
 \begin{rawhtml}
  <center><img src=../icons/planets0.gif alt="[planets in DrScheme]"></center>
 \end{rawhtml}}
\def\cards{
 \begin{rawhtml}
  <center><img src=../icons/cards.gif alt="[cards in DrScheme]"></center>
 \end{rawhtml}}
\def\ricetour{
 \begin{rawhtml}
  <center><img src=../icons/rice-tour.gif alt="[A Rice University Tour in DrScheme]"></center>
 \end{rawhtml}}
\def\morerice{
 \begin{rawhtml}
  <center><img src=../icons/rice-tour2.gif alt="[A Rice University Tour in DrScheme]"></center>
 \end{rawhtml}}
\def\intersection{
 \begin{rawhtml}
  <center><img src=../icons/traffic.gif alt="[An intersection]"></center>
 \end{rawhtml}}
\def\multiTL{
 \begin{rawhtml}
  <center><img src=../icons/multiple-traffic.gif alt="[Multiple Traffic Lights]"></center>
 \end{rawhtml}}
\def\drawfigpic{
 \begin{rawhtml}
 <table cellspacing=20 bgcolor=beige>
 <tr> <td align=left>  <img src=../icons/hang0.gif></td>
      <td align=center><img src=../icons/hang1.gif></td>
      <td align=right> <img src=../icons/hang2.gif></td>
 </tr>
 </table>
 \end{rawhtml}}
\def\lookuppic{
 \begin{rawhtml}
 <table cellspacing=20 bgcolor=beige>
 <tr> <td align=left>  <img src=../icons/lookup-gui0.gif></td>
      <td align=center><img src=../icons/lookup-gui1.gif></td>
      <td align=right> <img src=../icons/lookup-gui2.gif></td>
 </tr>
 </table>
 \end{rawhtml}}
\def\pbpic{
 \begin{rawhtml}
 <center><img src=../icons/phonebook.gif></td></center>
 \end{rawhtml}}
\def\hmguipic{
 \begin{rawhtml}
 <center><img src=../icons/hang-gui.gif></td></center>
 \end{rawhtml}}
\def\riotpic{
 \begin{rawhtml}
 <center><img src=../icons/riot.gif></td></center>
 \end{rawhtml}}
\def\guesspic{
 \begin{rawhtml}
 <table cellspacing=20 bgcolor=beige>
 <tr> <td align=left>  <img src=../icons/guess-gui0.gif></td>
      <td align=right> <img src=../icons/guess-gui1.gif></td>
 </tr>
 </table>
 \end{rawhtml}}
\def\guipad{
 \begin{rawhtml}
 <table cellspacing=20 bgcolor=beige>
 <tr> <td align=left>  <img src=../icons/vp.gif></td>
      <td align=right> <img src=../icons/cal.gif></td>
 </tr>
 </table>
 \end{rawhtml}}
\def\toytable{
 \begin{rawhtml}
 <center>
 <table border=1 bgcolor=beige>
 <tr>
   <td align=center valign=bottom>Item</td>
   <td align=center valign=bottom>Price</td> 
   <td align=center valign=bottom>Image</td>
 </tr>

 <tr>
   <td align=center valign=bottom>robot</td>
   <td align=center valign=bottom>29.95</td> 
   <td align=center valign=bottom><img src=../icons/robot.gif></td>
 </tr>

 <tr>
   <td align=center valign=bottom>robot</td>
   <td align=center valign=bottom>29.95</td> 
   <td align=center valign=bottom><img src=../icons/doll.gif></td>
 </tr>

 <tr>
   <td align=center valign=bottom>robot</td>
   <td align=center valign=bottom>29.95</td> 
   <td align=center valign=bottom><img src=../icons/rocket.gif></td>
 </tr>
 </table>
 </center>
 \end{rawhtml}}

\def\bezierpic{
 \begin{rawhtml}
 <table  cellspacing=20 bgcolor=beige>
 <tr> <td align=left><img src=../icons/bezier1.gif></td>
      <td align=center><img src=../icons/bezier2.gif></td>
      <td align=right><img src=../icons/bezier3.gif></td>
 </tr>
 </table>
 \end{rawhtml}}
\def\taskqpic{
 \begin{rawhtml}
 <table  cellspacing=20 bgcolor=beige>
 <tr> <td align=left><img src=../icons/taskq1.gif></td>
      <td align=center><img src=../icons/taskq2.gif></td>
      <td align=right><img src=../icons/taskq3.gif></td>
 </tr>
 </table>
 \end{rawhtml}}
\def\movingpic{
 \begin{rawhtml}
 <center>
 <table  cellspacing=20 bgcolor=beige>
   <tr><td align=center><img src=../icons/moving1.gif></td></tr>
   <tr><td align=center><img src=../icons/moving2.gif></td></tr>
 </table>
 </center>
 \end{rawhtml}}

\def\treepic{
 \begin{rawhtml}
 <table  cellspacing=20 bgcolor=beige>
 <tr> <td align=left><img src=../icons/tree1.gif></td>
      <td align=right><img src=../icons/tree2.gif></td>
 </tr>
 </table>
 \end{rawhtml}}
\def\siepic{
 \begin{rawhtml}
 <table  cellspacing=20 bgcolor=beige>
 <tr> <td align=left><img src=../icons/sie1.gif></td>
      <td align=center><img src=../icons/sie2.gif></td>
      <td align=right><img src=../icons/sie3.gif></td>
 </tr>
 </table>
 \end{rawhtml}}
\def\trafficpic{
 \begin{rawhtml}
 <table  cellspacing=20 bgcolor=beige>
 <tr> <td valign=top align=left><img src=../icons/next.gif></td>
      <td align=left><img src=../icons/green.gif></td>
      <td align=center><img src=../icons/yellow.gif></td>
      <td align=right><img src=../icons/red.gif></td>
 </tr>
 </table>
 \end{rawhtml}}
\def\siepicB{ \begin{rawhtml} <img align=center src=../icons/sie2.gif> \end{rawhtml} }
\def\datapic{ \begin{rawhtml} <img align=center src=../icons/data-collection.jpg> \end{rawhtml} }
\def\machinepic{\ \ \ {\htmladdimg{../icons/computer-pic.jpg}}\\}
\def\integratepic{\ \ \ {\htmladdimg{../icons/integrate.jpg}}\\}
\def\diffpic{\ \ \ {\htmladdimg{../icons/differentiate.jpg}}\\}
\endhtmlonly

%% Should work!
% \def\button#1{\textsf{\textbf{#1}}}
% \def\path#1{\textsf{\textbf{#1}}}
\newcommand{\path}[1]{{\tt\bf #1}}

\newcommand{\button}[1]{{\tt #1}}
\newcommand{\window}[1]{{\tt #1}}
\newcommand{\drmode}[1]{{\tt #1}}
\newcommand{\key}[1]{{\tt #1}}

% ---------------------------------------------------------------
% MACHINE INSTRUCTIONS
\def\iimmediate{\mbox{$\pm d_7 \ldots d_2$}}
\def\uimmediate{\mbox{$d_7 \ldots d_2$}}
 \newcommand{\jam}[8]{\hbox{$#1$ $#2$ $#3$ $#4$ $#5$ $#6$ $#7$ $#8$}} %  to .9in
 \newcommand{\jamIM}[2]{\hbox{$\iimmediate$ $#1$ $#2$}} %  to .9in

% ---------------------------------------------------------------
% SLATEX
\input{rs-config.tex}
\setspecialsymbol{natural-number}{{\bf N}} % this should be Bbb
\htmlonly
\def\anarrow{{}{\tt ->}{}}
\endhtmlonly

% ---------------------------------------------------------------
% EXERCISES

\newexercise{exercise}{Exercise}[subsection]

%% beginning of exercise block 
\def\exthick{1pt}
\newdimen\exwidth
\setlength\exwidth\textwidth
\advance\exwidth by -2pt

\def\exstrut{\rule{1pt}{5pt}}

\def\beginex{%
  \beginexline%
  \nopagebreak%
  \noindent{\large \bf Exercises}}

%% end of exercise and end of block
\def\beginexline{%
  \noindent\raisebox{-4pt}{\exstrut}%
	   \rule{\exwidth}{\exthick}%
	   \raisebox{-4pt}{\exstrut}\\[1ex]%
}

%% end of exercise 
\def\eoe{{\rule{3pt}{5pt}}}

%% end of exercise and end of block
\def\endex{\eoe\\\endexline}

\def\endexline{%
 \nopagebreak%
 \noindent\exstrut\rule{\exwidth}{\exthick}\exstrut}

\htmlonly
\def\beginexline{%
 \begin{rawhtml}
   <hr />
 \end{rawhtml}}

\def\endexline{
 \begin{rawhtml}
   <hr />
 \end{rawhtml}}
\endhtmlonly

% ---------------------------------------------------------------
% GUIDELINES

%% the width of guideline and dd boxes
\def\boxln{4.5in}

%% #1 : title of guideline (cap style)
%% #2 : body (rule)
\def\guideline#1#2{
\medskip
\fbox{\begin{minipage}{\boxln}
\centerline{\sc Guideline on #1}
\smallskip
#2
\end{minipage}}
\bigskip}

\htmlonly
\newcommand{\guideline}[2]{
  \begin{rawhtml}
    <table bgcolor=red align=center> 
     <tr><td><font align=center color=white size=+3>
  \end{rawhtml}
Guideline on #1
  \begin{rawhtml}
    </td></tr>
    <tr><td><font color=white><p>
  \end{rawhtml}
#2 
  \begin{rawhtml}
    </p>
   </td></tr>
  </table>
  \end{rawhtml}
}
\endhtmlonly

% ---------------------------------------------------------------
% DATA DEFINITIONS

%% #1 : text for dd
\long\def\dd#1{$$\mbox{\fbox{\begin{minipage}{\boxln}#1\end{minipage}}}$$}

%% THIS NEEDS FIXING FOR FINAL VERSION. 
\htmlonly
\newcommand{\dd}[1]{
\begin{rawhtml} 
<blockquote>
<table bgcolor="tan">
<tr><td>
\end{rawhtml}
#1 
\begin{rawhtml} 
</table>
</blockquote>
\end{rawhtml}
}
\endhtmlonly

% ---------------------------------------------------------------
% MISC

\def\alt{\,\,\vert\,\,}
\def\lenalt{\newdimen\p\setbox4=\hbox{$\alt$}\p=\wd4\kern\p}
\def\defined#1{{\sc {#1}}}
\def\defdd#1{{\sl {#1}}}

\def\geometry{In contrast to the usual Cartesian coordinate system, the
world of computer screens labels the upper-left corner of a window as the
origin.}

\def\bnf#1{\mbox{$\langle\mbox{\sl #1\/}\rangle$}}
\htmlonly
\def\bnf#1{\mbox{{\tt <{#1}>}}}
\endhtmlonly

\def\oldxaxis#1{%%
\begin{picture}(300,20)(.7,0)
\put(0,0){\line(1,0){300}}
\put(000,-5){\line(0,1){10}}
\put(-03,#1){\mbox{$0$}}
\put(025,-5){\line(0,1){10}}
\put(050,-5){\line(0,1){10}}
\put(075,-5){\line(0,1){10}}
\put(100,-5){\line(0,1){10}}
\put(125,-5){\line(0,1){10}}
\put(122,#1){\mbox{$5$}}
\put(150,-5){\line(0,1){10}}
\put(175,-5){\line(0,1){10}}
\put(200,-5){\line(0,1){10}}
\put(225,-5){\line(0,1){10}}
\put(250,-5){\line(0,1){10}}
\put(245,#1){\mbox{$10$}}
\put(275,-5){\line(0,1){10}}
\put(300,-5){\line(0,1){10}}
\end{picture}}

\def\oldyaxis{%%
\begin{picture}(20,200)(+.8,0) % x axis
\put(-5,000){\line(1,0){10}}
\put(-5,025){\line(1,0){10}}
\put(-15,46){\mbox{$5$}}
\put(-5,050){\line(1,0){10}}
\put(-5,075){\line(1,0){10}}
\put(-5,100){\line(1,0){10}}
\put(-5,125){\line(1,0){10}}
\put(-5,150){\line(1,0){10}}
\put(-5,175){\line(1,0){10}}
\put(-15,172){\mbox{$0$}}
\put(+0,175){\line(0,-1){175}}
\end{picture}}

\newcount\invcnt \invcnt=20
\def\intv#1#2#3{%
\invcnt=#1 \advance\invcnt by +4
\begin{picture}(\invcnt,5)\thicklines
\invcnt=#1 \advance\invcnt by -2
\put(0,0){\mbox{\Large#2}}
\put(2,2.5){\line(1,0){\invcnt}}
\put(2,3.0){\line(1,0){\invcnt}}
\put(2,3.5){\line(1,0){\invcnt}}
\invcnt=#1 \advance\invcnt by -3
\put(\invcnt,0){\mbox{\Large#3}}
\end{picture}}

%% ------------------------------------------------------------------------
%% NUMBER LINE

%% #i : what to put at 0 (relax, 0), ``5'' and ``10''
\def\xaxis#1#2#3{
\put(0,0){\line(1,0){300}}
\multiput(0,-5)(25,0){12}{\line(0,1){10}}
\put(-03,-18){\mbox{#1}}
\put(123,-18){\mbox{#2}}
\put(245,-18){\mbox{#3}}}

\def\yaxis{
\put(0,0){\line(0,-1){175}}
\multiput(-5,0)(0,-25){7}{\line(1,0){10}}
\put(-18,+03){\mbox{$0$}}
\put(-18,-123){\mbox{$5$}}}

%% the four ``brackets'' for intervals (OLD)
% \def\LOPEN#1{\put(#1,-4){\mbox{\hspace{-1pt}\LARGE(}}}
% \def\ROPEN#1{\put(#1,-4){\mbox{\hspace{-4.5pt}\LARGE)}}}
% \def\LCLSD#1{\put(#1,-4){\mbox{\hspace{-1.5pt}\LARGE[}}}
% \def\RCLSD#1{\put(#1,-4){\mbox{\hspace{-4pt}\LARGE]}}}

%% the four ``brackets'' for intervals
%% the maker is like apply so that dlatex can expaned properly 
%% #1 : a backet
%% #2 : a number
\def\mkbracket#1#2{\put(#2,-4){#1}}
\def\LOPEN{{\mbox{\hspace{-1pt}\LARGE(}}}
\def\ROPEN{{\mbox{\hspace{-4.5pt}\LARGE)}}}
\def\LCLSD{{\mbox{\hspace{-1.5pt}\LARGE[}}}
\def\RCLSD{{\mbox{\hspace{-4pt}\LARGE]}}}

%% intervals: 
%% bidirectional infinity (paras, see left/right infinity)
%% #4 : direction +/- 1
\def\INFI#1#2#3#4{\mkbracket{#3}{#1}\multiput(#1,1)(0,-.1){25}{\vector(#4,0){#2}}}

%% left-bounded infinity
%% to draw a big, bold arrow pointing right 
%% #1 : left boundary, 0, 25, 50, ...
%% #2 : length, multiples of 25
%% #3 : the left bracket (LOPEN or LCLSD)
\def\LINFI#1#2#3{\INFI{#1}{#2}{#3}{+1}}

%% right-bounded infinity
%% to draw a big, bold arrow pointing left
%% #1 : left boundary, 0, 25, 50, ...
%% #2 : length, multiples of 25
%% #3 : the left bracket (ROPEN or RCLSD)
\def\RINFI#1#2#3{\INFI{#1}{#2}{#3}{-1}}

%% bounded
%% to draw an interval between #1 and #2
%% #1 : left boundary, 0, 25, 50, ...
%% #2 : length, multiples of 25
%% #3 : the left bracket
%% #4 : the right bracket
\newcount\invcnt \invcnt=25
\def\INTVL#1#2#3#4{
 \mkbracket{#3}{#1} % left bracket
 \multiput(#1,1)(0,-.1){25}{\line(1,0){#2}} % lines 
 \invcnt=#1 \advance \invcnt by #2 
 \mkbracket{#4}{\invcnt} % right bracket
}

%% to draw a general numberline with things (#4) superimposed (intervals)
%% #1 - #3 : labels for x axis
%% #4 : \put's 
\def\gennumberln#1#2#3#4{
\par\smallskip
\hspace{.2cm}{\framebox[\numwidth]{
\begin{picture}(300,30)(0,-20)
\xaxis{#1}{#2}{#3}
%% ---
#4
\end{picture}}}}

%% to draw a numberline with things (#1) superimposed (intervals)
%% #1 : \put's 
\def\numberln#1{\gennumberln{$0$}{$5$}{$10$}{#1}}

%% CONSTANTS:
\newdimen\numwidth \numwidth=\textwidth \advance\numwidth by -40pt

%% ------------------------------------------------------------------------
%% PART
\def\newpart#1{
 \part{#1}
 \thispagestyle{empty}
}

\def\endpart{
%%\typeout{-- \thepage --}
 ~\newpage%
%%\typeout{-- \thepage --}
 \ifodd\thepage
%%   ``this is odd''
   \thispagestyle{empty}~\newpage ~\thispagestyle{empty}\newpage
 \else 
   \thispagestyle{empty}~\newpage
 \fi
}

%% ------------------------------------------------------------------------
%% DESIGN RECIPES in TABLES

\def\astrut{\raisebox{-10pt}{\mbox{}}}
\def\anex#1{\begin{minipage}[t]{1.0in}#1 \astrut\end{minipage}}
\def\ahin#1{\begin{minipage}[t]{1.6in}#1 \astrut\end{minipage}}
\def\agoal#1{\begin{minipage}[t]{1.6in}\raggedright #1 \astrut\end{minipage}}
\def\anactivity#1{\begin{minipage}[t]{2.2in}\raggedright {#1}\astrut\end{minipage}}
\def\act{$\bullet$\ }

\def\datagoal{\agoal{to formulate a data definition}}
\def\congoal{
  \agoal{to name the function;\\ 
	 to specify its classes of\\~~input data and its\\~~class of output data;\\
	 to describe its purpose;\\
	 to formulate a header
}}
\def\exgoal{\agoal{to characterize the input-\\ output relationship via examples}}
\def\tegoal{\agoal{to formulate an outline}}
\def\rlgoal{\agoal{to define the function}}
\def\testgoal{\agoal{to discover mistakes\\~~(``typos'' and logic)}}

\def\recipe#1#2#3#4#5#6#7{%% merged header and contract w/o moving paras
\begin{center}
\begin{tabular}{|l|l|l|}\hline
Phase            &  Goal 		& Activity \\ \hline\hline
\begin{minipage}[t]{.60in}
Data\\~~Analysis\\~~and Design
\end{minipage}  & \datagoal & #1   \\ \hline
\begin{minipage}[t]{.60in}
Contract\\
Purpose and\\
Header\\
\end{minipage}
	& \congoal  & #2 \\ \hline
Examples 	& \exgoal   & #3   \\ \hline
Template        & \tegoal   & #5   \\ \hline
Body	 	& \rlgoal   & #6   \\ \hline
Test		& \testgoal & #7   \\ \hline
\end{tabular}
\end{center}}

\def\shortrecipe#1#2#3#4#5{%%
\begin{center}
\begin{tabular}{|l|l|l|}\hline
Phase            &  Goal 		& Activity \\ \hline\hline
\begin{minipage}[t]{.60in}
Contract\\
Purpose and\\
Header\\
\end{minipage} 	& \congoal  & #1 \\ \hline
%Header   	& \hdgoal   & #3   \\ \hline
Examples 	& \exgoal   & #2   \\ \hline
Body	 	& \rlgoal   & #4   \\ \hline
Test		& \testgoal & #5 \\ \hline
\end{tabular}
\end{center}}

\def\asbefore#1{\multicolumn{2}{c|}{see figure~{#1}\raisebox{-10pt}{\rule{0pt}{15pt}}}}

% For dlatex:
\newcommand{\TabSpace}[0]{\makebox[1ex][l]{}} 
\htmlonly 
\newcommand{\TabSpace}[0]{\ } 

\def\beginfig{%
 \begin{rawhtml}
   <hr />
   <blockquote>
   <table bgcolor="beige">
   <tr><td>
 \end{rawhtml}
}
\def\caption#1{\center{Figure: #1}}

\def\endfig{%
 \begin{rawhtml}
   </td></tr>
   </table>
   </blockquote>
   <hr />
 \end{rawhtml}
}

\def\notinhtml{%
 \begin{rawhtml}
  <center>
    <font color=red>The figure is not yet translated into HTML.</font>
  </center>
 \end{rawhtml}
}

\endhtmlonly
