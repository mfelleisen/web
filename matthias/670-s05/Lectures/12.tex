\documentclass[11pt]{article}
\usepackage{a210}
\usepackage{palatino}
\usepackage{slatex}
\usepackage{epsf}
\usepackage{mztp}
\usepackage{html}

% ---------------------------------------------------------------
% special slatex symbols

\setspecialsymbol{fst}{{\it first\/}} % otherwise we get bad first

\setspecialsymbol{newline}{\relax}
\setspecialsymbol{anewline}{\relax}

\setspecialsymbol{<-ir}{$<_{ir}$}

\setspecialsymbol{Celsius->Fahrenheit}{{\arrowname{\it Celsius\/}{\it Fahrenheit\/}}}
\setspecialsymbol{Fahrenheit->Celsius}{{\arrowname{\it Fahrenheit\/}{\it Celsius\/}}}
\setspecialsymbol{dollar->euro}{{\arrowname{\it dollar\/}{\it euro\/}}}

\setspecialsymbol{inches->cm}{\arrowname{{\it inches\/}}{{\it cm\/}}}
\setspecialsymbol{feet->inches}{\arrowname{{\it feet\/}}{{\it inches\/}}}
\setspecialsymbol{yards->feet}{\arrowname{{\it yards\/}}{{\it feet\/}}}
\setspecialsymbol{rods->yards}{\arrowname{{\it rods\/}}{{\it yards\/}}}
\setspecialsymbol{furlongs->rods}{\arrowname{{\it furlongs\/}}{{\it rods\/}}}
\setspecialsymbol{miles->furlongs}{\arrowname{{\it miles\/}}{{\it furlongs\/}}}

\setspecialsymbol{feet->cm}{\arrowname{{\it feet\/}}{{\it cm\/}}}
\setspecialsymbol{yards->cm}{\arrowname{{\it yards\/}}{{\it cm\/}}}
\setspecialsymbol{rods->inches}{\arrowname{{\it rods\/}}{{\it inches\/}}}
\setspecialsymbol{miles->feet}{\arrowname{{\it miles\/}}{{\it feet\/}}}

\setspecialsymbol{time->seconds}{\arrowname{\it time\/}{\it seconds\/}}

\setspecialsymbol{hours->wages}{\arrowname{\it hours\/}{\it wages\/}} 

\setspecialsymbol{boxed:rel-op}{\fbox{{\it rel-op\/}}}
\setspecialsymbol{boxed:op1}{\fbox{{\it op1\/}}}
\setspecialsymbol{boxed:op2}{\fbox{{\it op2\/}}}
\setspecialsymbol{boxed:doll}{\fbox{{\sf 'doll}}}
\setspecialsymbol{boxed:car}{\fbox{{\sf 'car}}}
\setspecialsymbol{boxed:s}{\fbox{$s$}}
\setspecialsymbol{boxed:lt}{\fbox{$<$}}
\setspecialsymbol{boxed:gt}{\fbox{$>$}}
\setspecialsymbol{boxedRO}{\fbox{{\it rel-op}}}
\setspecialsymbol{boxed:ltir}{\fbox{$<_{ir}$}}
\setspecialsymbol{<-ir}{$<_{ir}$}
\setspecialsymbol{C->F}{\arrowname{\it C\/}{\it F\/}}
\setspecialsymbol{boxed:C->F}{\fbox{{\arrowname{\it C\/}{\it F\/}}}}
\setspecialsymbol{boxed:IR-name}{\fbox{{\it IR-name}}}
\setspecialsymbol{boxed:f}{\fbox{{\it f}}}

\setspecialsymbol{number->string}{{\arrowname{{\it number\/}}{{\it string\/}}}}
\setspecialsymbol{string->symbol}{\arrowname{{\it string\/}}{{\it symbol\/}}}
\setspecialsymbol{string->number}{\arrowname{{\it string\/}}{{\it number\/}}}
\setspecialsymbol{pad->gui}{\arrowname{{\it pad\/}}{{\it gui\/}}}

\setspecialsymbol{hours->wages}{\arrowname{\it hours\/}{\it wages\/}} 
\setspecialsymbol{hours->wages2}{\arrowname{\it hours\/}{\it wages2\/}} 
\setspecialsymbol{hours->wages3}{\arrowname{\it hours\/}{\it wages3\/}} 

\setspecialsymbol{test-hours->wages}{\arrowname{\it test-hours\/}{\it wages\/}} 

\setspecialsymbol{file->list-of-lines}{\arrowname{{\it file\/}}{{\it list-of-lines}}}
\setspecialsymbol{file->list-of-checks}{\arrowname{{\it file\/}}{{\it list-of-checks}}}


\setconstant{
make-posn posn-x posn-y set-posn-x! set-posn-y! posn?
make-aaa aaa-xx aaa-yy set-aaa-xx! set-aaa-yy! aaa?
make-bbb bbb-zz bbb-ww set-bbb-zz! set-bbb-ww! bbb?
make-node node-name node-to set-node-name! set-node-to! node?
  node-visited set-node-visited!
make-person person-name person-social person-father person-mother person-children
  set-person-father! set-person-mother! set-person-children!
  person?
make-entry entry-name entry-number set-entry-name! set-entry-number! entry? 
make-cheerleader
make-circle circle-center circle-radius circle? 
make-square square-nw square-length square? 
make-hand hand-rank hand-suit hand-next hand? set-hand-next! 
}


\setconstant{
make-mail
make-file file-name file-size file-content file?
make-dir dir-name dir-dirs dir-files file-content dir?
make-node node-ssn node-name node-left node-right node?
make-child child-father child-mother child-name child-date child-eyes child?
make-posn posn-x posn-y posn?
make-add add-left add-right add? 
make-mul mul-left mul-right mul? 
make-prim prim-name prim-left prim-right prim? 
make-parent parent-children parent-name parent-eyes parent-date parent?
make-roman roman-grp roman?
make-italics italics-grp italics?
make-bold-face bold-face-grp bold-face?
make-wp wp-header wp-body wp?
make-phone-record phone-record-name phone-record-number phone-record?
}


\setconstant{
make-vec vec-x vec-y vec?
make-posn posn-x posn-y posn?
make-point point-x point-y point-z point? 
make-movie movie-title movie-producer movie? 
make-star star-name star-first star-instrument star-sales star? 
make-student student-last student-first student-teacher
make-airplane airplane? 
make-circle circle-center circle-radius circle? 
make-square square-nw square-length square? 
make-word make-time
make-entry entry-name entry-zip entry-phone
make-ball ball-posn
}

\setconstant{
make-check-box
make-radio-box
make-vec vec-x vec-y vec?
make-posn posn-x posn-y posn?
make-point point-x point-y point-z point? 
make-movie movie-title movie-producer movie? 
make-star star-last star-first star-instrument star-sales star? 
make-student student-last student-first student-teacher
make-airplane airplane? 
make-circle circle-center circle-radius circle? 
make-square square-nw square-length square? 
make-word make-time
make-entry entry-name entry-zip entry-phone
make-ball ball-posn
make-pair pair-left pair-right pair? 
make-ir ir-name ir-price ir?
make-er er-name er-rate er?
}


\setconstant{
make-posn posn-x posn-y posn? 
make-ball ball-x ball-y ball-delta-x ball-delta-y ball? 
make-rr
}


\setconstant{
make-hide hide-it hide?
make-loc loc-con loc?
}

% ---------------------------------------------------------------
% for dependence.tex 
\newcounter{foops}


% ---------------------------------------------------------------
% for prologue and epilogue

%% #1: quote
%% #2: title 
%% #3: toc line 
\def\ppart#1{\noindent {\Large #1}\\
  \noindent \rule{\textwidth}{4pt}\\
  \vskip 20ex}

\def\psect#1{\subsection*{#1}}

\htmlonly
\def\ppart#1{\part{#1}}
\def\psect#1{\subsection{#1}}
\endhtmlonly

% ---------------------------------------------------------------
% html coversion of little things
\def\ul#1{\underline{#1}}

\htmlonly
\def\ul#1{%
 \begin{rawhtml}<u>\end{rawhtml}#1\begin{rawhtml}</u>\end{rawhtml}%
}
\endhtmlonly

\def\anote#1{\medskip\noindent{\bf #1}:\ }

% ---------------------------------------------------------------
%% engl

\def\hints{\noindent{\bf Hints:} }
\def\hint{\noindent{\bf Hint:} }

% ---------------------------------------------------------------
% FIGURES
\def\beginfig{\begin{figure*}[htbp]{\noindent\hrulefill\par}\small}
\def\endfig{{\noindent\hrulefill\par}\end{figure*}}

\renewcommand{\topfraction}{1}
\renewcommand{\dbltopfraction}{1}
\renewcommand{\bottomfraction}{1}
\renewcommand{\textfraction}{0}

\renewcommand{\floatpagefraction}{.90} % for one-column
\renewcommand{\dblfloatpagefraction}{.99} % for two-.

% ---------------------------------------------------------------
% LABELS

\pagestyle{headings}

\def\intermezzo#1#2{%
 \refstepcounter{section}
 \setcounter{subsection}{0} 
 \section*{#1:~#2}
 \addcontentsline{toc}{part}{#1:~#2}
 \markboth{{\footnotesize #1}}{{\footnotesize #2}}
 }
\def\subintermezzo#1{%
 \refstepcounter{subsection}\subsection*{#1} 
 \addcontentsline{toc}{subsection}{#1}
 \setcounter{exercise}{0}}

\htmlonly
\def\intermezzo#1#2{%
 \part{#1~#2}
 \section{#1~#2}
 }
\def\subintermezzo#1{%
 \subsection{#1}}
\endhtmlonly

% ---------------------------------------------------------------
% NOTES, CROSS REFERENCES

% #1: text for marginal para
\def\mp#1{\marginpar{\footnotesize \raggedright #1}}

\def\plt#1{\marginpar{\footnotesize \raggedright {\sc PLT}: #1}}

\def\teacher{\epsfbox{icons/teacher1.ps}}
\def\drscheme{\epsfxsize=.5in\epsfbox{icons/plt.ps}}

\def\drstuff#1{{\footnotesize
{\sf
\begin{flushright}
#1
\end{flushright}}}}

\def\drhint#1#2{% 
 \marginpar{\hfill\drscheme}
 \marginpar{\drstuff{#2}}}

\long\def\note#1{\relax}    %%{\marginpar[\hfill\teacher]{\teacher}}

\def\hand{\epsfbox{icons/hand.right.ps}}
\def\solution#1{\relax}

\htmlonly 
\def\plt#1{\relax}
\def\teacher{\htmladdimg{../icons/teacher1.gif}}
\def\drscheme{\htmladdimg{../icons/plt.gif}}
\def\drhint#1#2{\htmlref{\drscheme}{#1}}

\long\def\note#1{\\ \teacher \\ }  %% Don't forget the space!!!

\def\hand{
\begin{rawhtml}
<img src="../icons/hand.right.gif">
\end{rawhtml}
}

\def\solution#1{\htmladdnormallink{Solution}{../Solutions/#1.html}}
% {\hand}
\endhtmlonly

% ---------------------------------------------------------------
\def\achild#1#2#3{\begin{picture}(75,30)
\put(00,00){\line(+1,+0){70}}
\put(70,00){\line(+0,+1){30}}
\put(70,30){\line(-1,+0){70}}
\put(00,30){\line(+0,-1){30}}
\put(00,19){\mbox{~#1~~(#2)}}
\put(00,05){\mbox{~Eyes:~~ #3}}
\end{picture}}

% ---------------------------------------------------------------
% PS FILES FOR CROSS REFERENCES
\def\spades{\epsfbox{icons/spades.ps}}
\def\clubs{\epsfbox{icons/clubs.ps}}

\def\planets{\centerline{\epsfxsize=5in\epsfbox{icons/planets0.ps}}}
\def\cards{\centerline{\epsfxsize=5in\epsfbox{icons/cards.ps}}}
\def\ricetour{\centerline{\epsfxsize=5in\epsfbox{icons/rice-tour.ps}}}
\def\morerice{\centerline{\epsfxsize=5in\epsfbox{icons/rice-tour2.ps}}}
\def\intersection{\centerline{\epsfxsize=1in\epsfbox{icons/traffic.ps}}}
\def\multiTL{\centerline{\epsfysize=1in\epsfbox{icons/multiple-traffic.ps}}}

\def\drawfigpic{
 \begin{center}
 \hbox to \textwidth{
  \hfill
  \epsfxsize=1.5in \epsfbox{icons/hang0.ps} 
  \hfill
  \epsfxsize=1.5in \epsfbox{icons/hang1.ps}
  \hfill
  \epsfxsize=1.5in \epsfbox{icons/hang2.ps}
  \hfill}
  \end{center}}
\def\lookuppic{
 \begin{center}
 \hbox to \textwidth{
  \hfill
  \epsfxsize=1.5in \epsfbox{icons/lookup-gui0.ps} 
  \hfill
  \epsfxsize=1.5in \epsfbox{icons/lookup-gui1.ps}
  \hfill
  \epsfxsize=1.5in \epsfbox{icons/lookup-gui2.ps}
  \hfill}
  \end{center}}
\def\pbpic{\smallskip
 \centerline{\epsfxsize=2.5in \epsfbox{icons/phonebook.ps}}
}

\def\hmguipic{\smallskip
 \centerline{\epsfxsize=2.5in \epsfbox{icons/hang-gui.ps}}
}

\def\riotpic{\smallskip
 \centerline{\epsfxsize=1.in \epsfbox{icons/riot.ps}}
}

\def\guipad{
 \begin{center}
 \hbox to \textwidth{
  \hfill
  \epsfysize=1.5in 
  \epsfbox{icons/vp.ps} 
  \hspace{2in}
  \epsfysize=1.5in 
  \epsfbox{icons/cal.ps}
  \hfill}
  \end{center}}

\def\toytable{
 \begin{center}
 \begin{tabular}{l|r|c}\hline
  \hbox to 1.2in{Symbol} & \hbox to 1.2in{Price} & \hbox to 1.2in{Image} \\ \hline 
  && \\
  robot  & 11.95 & \epsfbox{icons/robot.ps} \\
  && \\
   doll  & 19.95 & \epsfbox{icons/doll.ps} \\
  && \\
 rocket  & 29.95 & \epsfbox{icons/rocket.ps} \\
  \vdots & \vdots & \vdots \\
  \end{tabular}
 \end{center}}

\def\bezierpic{
 \begin{center}
 \hbox to \textwidth{
  \hfill
  \epsfxsize=1.5in \epsfbox{icons/bezier1.ps} 
  \hfill
  \epsfxsize=1.5in \epsfbox{icons/bezier2.ps}
  \hfill
  \epsfxsize=1.5in \epsfbox{icons/bezier3.ps}
  \hfill}
  \end{center}}
\def\taskqpic{
 \begin{center}
 \hbox to \textwidth{
  \hfill
  \epsfxsize=1.5in \epsfbox{icons/taskq1.ps} 
  \hfill
  \epsfxsize=1.5in \epsfbox{icons/taskq2.ps}
  \hfill
  \epsfxsize=1.5in \epsfbox{icons/taskq3.ps}
  \hfill}
  \end{center}}
\def\movingpic{ %% DONT COPY!!!!!!!!!
  \smallskip
  \centerline{\epsfxsize=\textwidth\epsfbox{icons/moving1.ps}}
  \smallskip
  \centerline{\epsfxsize=\textwidth\epsfbox{icons/moving2.ps}}}
\def\treepic{
 \begin{center}
 \hbox to \textwidth{
  \hfill
  {\epsfxsize=1.5in\epsfbox{icons/tree1.ps}}
  \hfill
  {\epsfxsize=1.5in\epsfbox{icons/tree2.ps}}
  \hfill}
 \end{center}}
\def\siepic{
 \begin{center}
 \hbox to \textwidth{  \hfill
  \epsfxsize=1.5in \epsfbox{icons/sie1.ps} 
  \hfill
  \epsfxsize=1.5in \epsfbox{icons/sie2.ps} 
  \hfill
  \epsfxsize=1.5in \epsfbox{icons/sie3.ps}
  \hfill}
  \end{center}}
\def\trafficpic{
 \begin{center}
 \hbox to \textwidth{  \hfill
  \epsfysize=.5in 
  \epsfbox{icons/next.ps} 
  \hfill
  \epsfysize=1in 
  \epsfbox{icons/green.ps} 
  \hfill
  \epsfysize=1in 
  \epsfbox{icons/yellow.ps} 
  \hfill
  \epsfysize=1in 
  \epsfbox{icons/red.ps}
  \hfill}
  \end{center}}

\def\siepicB{
  \begin{center}
  \hbox to \textwidth{\hfill \epsfxsize=1.5in \epsfbox{icons/sie2.ps} \hfill}
  \end{center}}
\def\diffpic{\centerline{\epsfxsize=4in \epsfbox{icons/differentiate.eps}}}
\def\integratepic{\centerline{\epsfxsize=4in \epsfbox{icons/integrate.eps}}}
\def\machinepic{\centerline{\epsfxsize=4in \epsfbox{icons/computer-pic.ps}}}

\htmlonly 
\def\spades{
  \begin{rawhtml}
  <img src=../icons/spades.gif alt="[spades]">
  \end{rawhtml}}
\def\clubs{
  \begin{rawhtml}
  <img src=../icons/clubs.gif alt="[clubs]">
  \end{rawhtml}}

\def\planets{
 \begin{rawhtml}
  <center><img src=../icons/planets0.gif alt="[planets in DrScheme]"></center>
 \end{rawhtml}}
\def\cards{
 \begin{rawhtml}
  <center><img src=../icons/cards.gif alt="[cards in DrScheme]"></center>
 \end{rawhtml}}
\def\ricetour{
 \begin{rawhtml}
  <center><img src=../icons/rice-tour.gif alt="[A Rice University Tour in DrScheme]"></center>
 \end{rawhtml}}
\def\morerice{
 \begin{rawhtml}
  <center><img src=../icons/rice-tour2.gif alt="[A Rice University Tour in DrScheme]"></center>
 \end{rawhtml}}
\def\intersection{
 \begin{rawhtml}
  <center><img src=../icons/traffic.gif alt="[An intersection]"></center>
 \end{rawhtml}}
\def\multiTL{
 \begin{rawhtml}
  <center><img src=../icons/multiple-traffic.gif alt="[Multiple Traffic Lights]"></center>
 \end{rawhtml}}
\def\drawfigpic{
 \begin{rawhtml}
 <table cellspacing=20 bgcolor=beige>
 <tr> <td align=left>  <img src=../icons/hang0.gif></td>
      <td align=center><img src=../icons/hang1.gif></td>
      <td align=right> <img src=../icons/hang2.gif></td>
 </tr>
 </table>
 \end{rawhtml}}
\def\lookuppic{
 \begin{rawhtml}
 <table cellspacing=20 bgcolor=beige>
 <tr> <td align=left>  <img src=../icons/lookup-gui0.gif></td>
      <td align=center><img src=../icons/lookup-gui1.gif></td>
      <td align=right> <img src=../icons/lookup-gui2.gif></td>
 </tr>
 </table>
 \end{rawhtml}}
\def\pbpic{
 \begin{rawhtml}
 <center><img src=../icons/phonebook.gif></td></center>
 \end{rawhtml}}
\def\hmguipic{
 \begin{rawhtml}
 <center><img src=../icons/hang-gui.gif></td></center>
 \end{rawhtml}}
\def\riotpic{
 \begin{rawhtml}
 <center><img src=../icons/riot.gif></td></center>
 \end{rawhtml}}
\def\guesspic{
 \begin{rawhtml}
 <table cellspacing=20 bgcolor=beige>
 <tr> <td align=left>  <img src=../icons/guess-gui0.gif></td>
      <td align=right> <img src=../icons/guess-gui1.gif></td>
 </tr>
 </table>
 \end{rawhtml}}
\def\guipad{
 \begin{rawhtml}
 <table cellspacing=20 bgcolor=beige>
 <tr> <td align=left>  <img src=../icons/vp.gif></td>
      <td align=right> <img src=../icons/cal.gif></td>
 </tr>
 </table>
 \end{rawhtml}}
\def\toytable{
 \begin{rawhtml}
 <center>
 <table border=1 bgcolor=beige>
 <tr>
   <td align=center valign=bottom>Item</td>
   <td align=center valign=bottom>Price</td> 
   <td align=center valign=bottom>Image</td>
 </tr>

 <tr>
   <td align=center valign=bottom>robot</td>
   <td align=center valign=bottom>29.95</td> 
   <td align=center valign=bottom><img src=../icons/robot.gif></td>
 </tr>

 <tr>
   <td align=center valign=bottom>robot</td>
   <td align=center valign=bottom>29.95</td> 
   <td align=center valign=bottom><img src=../icons/doll.gif></td>
 </tr>

 <tr>
   <td align=center valign=bottom>robot</td>
   <td align=center valign=bottom>29.95</td> 
   <td align=center valign=bottom><img src=../icons/rocket.gif></td>
 </tr>
 </table>
 </center>
 \end{rawhtml}}

\def\bezierpic{
 \begin{rawhtml}
 <table  cellspacing=20 bgcolor=beige>
 <tr> <td align=left><img src=../icons/bezier1.gif></td>
      <td align=center><img src=../icons/bezier2.gif></td>
      <td align=right><img src=../icons/bezier3.gif></td>
 </tr>
 </table>
 \end{rawhtml}}
\def\taskqpic{
 \begin{rawhtml}
 <table  cellspacing=20 bgcolor=beige>
 <tr> <td align=left><img src=../icons/taskq1.gif></td>
      <td align=center><img src=../icons/taskq2.gif></td>
      <td align=right><img src=../icons/taskq3.gif></td>
 </tr>
 </table>
 \end{rawhtml}}
\def\movingpic{
 \begin{rawhtml}
 <center>
 <table  cellspacing=20 bgcolor=beige>
   <tr><td align=center><img src=../icons/moving1.gif></td></tr>
   <tr><td align=center><img src=../icons/moving2.gif></td></tr>
 </table>
 </center>
 \end{rawhtml}}

\def\treepic{
 \begin{rawhtml}
 <table  cellspacing=20 bgcolor=beige>
 <tr> <td align=left><img src=../icons/tree1.gif></td>
      <td align=right><img src=../icons/tree2.gif></td>
 </tr>
 </table>
 \end{rawhtml}}
\def\siepic{
 \begin{rawhtml}
 <table  cellspacing=20 bgcolor=beige>
 <tr> <td align=left><img src=../icons/sie1.gif></td>
      <td align=center><img src=../icons/sie2.gif></td>
      <td align=right><img src=../icons/sie3.gif></td>
 </tr>
 </table>
 \end{rawhtml}}
\def\trafficpic{
 \begin{rawhtml}
 <table  cellspacing=20 bgcolor=beige>
 <tr> <td valign=top align=left><img src=../icons/next.gif></td>
      <td align=left><img src=../icons/green.gif></td>
      <td align=center><img src=../icons/yellow.gif></td>
      <td align=right><img src=../icons/red.gif></td>
 </tr>
 </table>
 \end{rawhtml}}
\def\siepicB{ \begin{rawhtml} <img align=center src=../icons/sie2.gif> \end{rawhtml} }
\def\datapic{ \begin{rawhtml} <img align=center src=../icons/data-collection.jpg> \end{rawhtml} }
\def\machinepic{\ \ \ {\htmladdimg{../icons/computer-pic.jpg}}\\}
\def\integratepic{\ \ \ {\htmladdimg{../icons/integrate.jpg}}\\}
\def\diffpic{\ \ \ {\htmladdimg{../icons/differentiate.jpg}}\\}
\endhtmlonly

%% Should work!
% \def\button#1{\textsf{\textbf{#1}}}
% \def\path#1{\textsf{\textbf{#1}}}
\newcommand{\path}[1]{{\tt\bf #1}}

\newcommand{\button}[1]{{\tt #1}}
\newcommand{\window}[1]{{\tt #1}}
\newcommand{\drmode}[1]{{\tt #1}}
\newcommand{\key}[1]{{\tt #1}}

% ---------------------------------------------------------------
% MACHINE INSTRUCTIONS
\def\iimmediate{\mbox{$\pm d_7 \ldots d_2$}}
\def\uimmediate{\mbox{$d_7 \ldots d_2$}}
 \newcommand{\jam}[8]{\hbox{$#1$ $#2$ $#3$ $#4$ $#5$ $#6$ $#7$ $#8$}} %  to .9in
 \newcommand{\jamIM}[2]{\hbox{$\iimmediate$ $#1$ $#2$}} %  to .9in

% ---------------------------------------------------------------
% SLATEX
\input{rs-config.tex}
\setspecialsymbol{natural-number}{{\bf N}} % this should be Bbb
\htmlonly
\def\anarrow{{}{\tt ->}{}}
\endhtmlonly

% ---------------------------------------------------------------
% EXERCISES

\newexercise{exercise}{Exercise}[subsection]

%% beginning of exercise block 
\def\exthick{1pt}
\newdimen\exwidth
\setlength\exwidth\textwidth
\advance\exwidth by -2pt

\def\exstrut{\rule{1pt}{5pt}}

\def\beginex{%
  \beginexline%
  \nopagebreak%
  \noindent{\large \bf Exercises}}

%% end of exercise and end of block
\def\beginexline{%
  \noindent\raisebox{-4pt}{\exstrut}%
	   \rule{\exwidth}{\exthick}%
	   \raisebox{-4pt}{\exstrut}\\[1ex]%
}

%% end of exercise 
\def\eoe{{\rule{3pt}{5pt}}}

%% end of exercise and end of block
\def\endex{\eoe\\\endexline}

\def\endexline{%
 \nopagebreak%
 \noindent\exstrut\rule{\exwidth}{\exthick}\exstrut}

\htmlonly
\def\beginexline{%
 \begin{rawhtml}
   <hr />
 \end{rawhtml}}

\def\endexline{
 \begin{rawhtml}
   <hr />
 \end{rawhtml}}
\endhtmlonly

% ---------------------------------------------------------------
% GUIDELINES

%% the width of guideline and dd boxes
\def\boxln{4.5in}

%% #1 : title of guideline (cap style)
%% #2 : body (rule)
\def\guideline#1#2{
\medskip
\fbox{\begin{minipage}{\boxln}
\centerline{\sc Guideline on #1}
\smallskip
#2
\end{minipage}}
\bigskip}

\htmlonly
\newcommand{\guideline}[2]{
  \begin{rawhtml}
    <table bgcolor=red align=center> 
     <tr><td><font align=center color=white size=+3>
  \end{rawhtml}
Guideline on #1
  \begin{rawhtml}
    </td></tr>
    <tr><td><font color=white><p>
  \end{rawhtml}
#2 
  \begin{rawhtml}
    </p>
   </td></tr>
  </table>
  \end{rawhtml}
}
\endhtmlonly

% ---------------------------------------------------------------
% DATA DEFINITIONS

%% #1 : text for dd
\long\def\dd#1{$$\mbox{\fbox{\begin{minipage}{\boxln}#1\end{minipage}}}$$}

%% THIS NEEDS FIXING FOR FINAL VERSION. 
\htmlonly
\newcommand{\dd}[1]{
\begin{rawhtml} 
<blockquote>
<table bgcolor="tan">
<tr><td>
\end{rawhtml}
#1 
\begin{rawhtml} 
</table>
</blockquote>
\end{rawhtml}
}
\endhtmlonly

% ---------------------------------------------------------------
% MISC

\def\alt{\,\,\vert\,\,}
\def\lenalt{\newdimen\p\setbox4=\hbox{$\alt$}\p=\wd4\kern\p}
\def\defined#1{{\sc {#1}}}
\def\defdd#1{{\sl {#1}}}

\def\geometry{In contrast to the usual Cartesian coordinate system, the
world of computer screens labels the upper-left corner of a window as the
origin.}

\def\bnf#1{\mbox{$\langle\mbox{\sl #1\/}\rangle$}}
\htmlonly
\def\bnf#1{\mbox{{\tt <{#1}>}}}
\endhtmlonly

\def\oldxaxis#1{%%
\begin{picture}(300,20)(.7,0)
\put(0,0){\line(1,0){300}}
\put(000,-5){\line(0,1){10}}
\put(-03,#1){\mbox{$0$}}
\put(025,-5){\line(0,1){10}}
\put(050,-5){\line(0,1){10}}
\put(075,-5){\line(0,1){10}}
\put(100,-5){\line(0,1){10}}
\put(125,-5){\line(0,1){10}}
\put(122,#1){\mbox{$5$}}
\put(150,-5){\line(0,1){10}}
\put(175,-5){\line(0,1){10}}
\put(200,-5){\line(0,1){10}}
\put(225,-5){\line(0,1){10}}
\put(250,-5){\line(0,1){10}}
\put(245,#1){\mbox{$10$}}
\put(275,-5){\line(0,1){10}}
\put(300,-5){\line(0,1){10}}
\end{picture}}

\def\oldyaxis{%%
\begin{picture}(20,200)(+.8,0) % x axis
\put(-5,000){\line(1,0){10}}
\put(-5,025){\line(1,0){10}}
\put(-15,46){\mbox{$5$}}
\put(-5,050){\line(1,0){10}}
\put(-5,075){\line(1,0){10}}
\put(-5,100){\line(1,0){10}}
\put(-5,125){\line(1,0){10}}
\put(-5,150){\line(1,0){10}}
\put(-5,175){\line(1,0){10}}
\put(-15,172){\mbox{$0$}}
\put(+0,175){\line(0,-1){175}}
\end{picture}}

\newcount\invcnt \invcnt=20
\def\intv#1#2#3{%
\invcnt=#1 \advance\invcnt by +4
\begin{picture}(\invcnt,5)\thicklines
\invcnt=#1 \advance\invcnt by -2
\put(0,0){\mbox{\Large#2}}
\put(2,2.5){\line(1,0){\invcnt}}
\put(2,3.0){\line(1,0){\invcnt}}
\put(2,3.5){\line(1,0){\invcnt}}
\invcnt=#1 \advance\invcnt by -3
\put(\invcnt,0){\mbox{\Large#3}}
\end{picture}}

%% ------------------------------------------------------------------------
%% NUMBER LINE

%% #i : what to put at 0 (relax, 0), ``5'' and ``10''
\def\xaxis#1#2#3{
\put(0,0){\line(1,0){300}}
\multiput(0,-5)(25,0){12}{\line(0,1){10}}
\put(-03,-18){\mbox{#1}}
\put(123,-18){\mbox{#2}}
\put(245,-18){\mbox{#3}}}

\def\yaxis{
\put(0,0){\line(0,-1){175}}
\multiput(-5,0)(0,-25){7}{\line(1,0){10}}
\put(-18,+03){\mbox{$0$}}
\put(-18,-123){\mbox{$5$}}}

%% the four ``brackets'' for intervals (OLD)
% \def\LOPEN#1{\put(#1,-4){\mbox{\hspace{-1pt}\LARGE(}}}
% \def\ROPEN#1{\put(#1,-4){\mbox{\hspace{-4.5pt}\LARGE)}}}
% \def\LCLSD#1{\put(#1,-4){\mbox{\hspace{-1.5pt}\LARGE[}}}
% \def\RCLSD#1{\put(#1,-4){\mbox{\hspace{-4pt}\LARGE]}}}

%% the four ``brackets'' for intervals
%% the maker is like apply so that dlatex can expaned properly 
%% #1 : a backet
%% #2 : a number
\def\mkbracket#1#2{\put(#2,-4){#1}}
\def\LOPEN{{\mbox{\hspace{-1pt}\LARGE(}}}
\def\ROPEN{{\mbox{\hspace{-4.5pt}\LARGE)}}}
\def\LCLSD{{\mbox{\hspace{-1.5pt}\LARGE[}}}
\def\RCLSD{{\mbox{\hspace{-4pt}\LARGE]}}}

%% intervals: 
%% bidirectional infinity (paras, see left/right infinity)
%% #4 : direction +/- 1
\def\INFI#1#2#3#4{\mkbracket{#3}{#1}\multiput(#1,1)(0,-.1){25}{\vector(#4,0){#2}}}

%% left-bounded infinity
%% to draw a big, bold arrow pointing right 
%% #1 : left boundary, 0, 25, 50, ...
%% #2 : length, multiples of 25
%% #3 : the left bracket (LOPEN or LCLSD)
\def\LINFI#1#2#3{\INFI{#1}{#2}{#3}{+1}}

%% right-bounded infinity
%% to draw a big, bold arrow pointing left
%% #1 : left boundary, 0, 25, 50, ...
%% #2 : length, multiples of 25
%% #3 : the left bracket (ROPEN or RCLSD)
\def\RINFI#1#2#3{\INFI{#1}{#2}{#3}{-1}}

%% bounded
%% to draw an interval between #1 and #2
%% #1 : left boundary, 0, 25, 50, ...
%% #2 : length, multiples of 25
%% #3 : the left bracket
%% #4 : the right bracket
\newcount\invcnt \invcnt=25
\def\INTVL#1#2#3#4{
 \mkbracket{#3}{#1} % left bracket
 \multiput(#1,1)(0,-.1){25}{\line(1,0){#2}} % lines 
 \invcnt=#1 \advance \invcnt by #2 
 \mkbracket{#4}{\invcnt} % right bracket
}

%% to draw a general numberline with things (#4) superimposed (intervals)
%% #1 - #3 : labels for x axis
%% #4 : \put's 
\def\gennumberln#1#2#3#4{
\par\smallskip
\hspace{.2cm}{\framebox[\numwidth]{
\begin{picture}(300,30)(0,-20)
\xaxis{#1}{#2}{#3}
%% ---
#4
\end{picture}}}}

%% to draw a numberline with things (#1) superimposed (intervals)
%% #1 : \put's 
\def\numberln#1{\gennumberln{$0$}{$5$}{$10$}{#1}}

%% CONSTANTS:
\newdimen\numwidth \numwidth=\textwidth \advance\numwidth by -40pt

%% ------------------------------------------------------------------------
%% PART
\def\newpart#1{
 \part{#1}
 \thispagestyle{empty}
}

\def\endpart{
%%\typeout{-- \thepage --}
 ~\newpage%
%%\typeout{-- \thepage --}
 \ifodd\thepage
%%   ``this is odd''
   \thispagestyle{empty}~\newpage ~\thispagestyle{empty}\newpage
 \else 
   \thispagestyle{empty}~\newpage
 \fi
}

%% ------------------------------------------------------------------------
%% DESIGN RECIPES in TABLES

\def\astrut{\raisebox{-10pt}{\mbox{}}}
\def\anex#1{\begin{minipage}[t]{1.0in}#1 \astrut\end{minipage}}
\def\ahin#1{\begin{minipage}[t]{1.6in}#1 \astrut\end{minipage}}
\def\agoal#1{\begin{minipage}[t]{1.6in}\raggedright #1 \astrut\end{minipage}}
\def\anactivity#1{\begin{minipage}[t]{2.2in}\raggedright {#1}\astrut\end{minipage}}
\def\act{$\bullet$\ }

\def\datagoal{\agoal{to formulate a data definition}}
\def\congoal{
  \agoal{to name the function;\\ 
	 to specify its classes of\\~~input data and its\\~~class of output data;\\
	 to describe its purpose;\\
	 to formulate a header
}}
\def\exgoal{\agoal{to characterize the input-\\ output relationship via examples}}
\def\tegoal{\agoal{to formulate an outline}}
\def\rlgoal{\agoal{to define the function}}
\def\testgoal{\agoal{to discover mistakes\\~~(``typos'' and logic)}}

\def\recipe#1#2#3#4#5#6#7{%% merged header and contract w/o moving paras
\begin{center}
\begin{tabular}{|l|l|l|}\hline
Phase            &  Goal 		& Activity \\ \hline\hline
\begin{minipage}[t]{.60in}
Data\\~~Analysis\\~~and Design
\end{minipage}  & \datagoal & #1   \\ \hline
\begin{minipage}[t]{.60in}
Contract\\
Purpose and\\
Header\\
\end{minipage}
	& \congoal  & #2 \\ \hline
Examples 	& \exgoal   & #3   \\ \hline
Template        & \tegoal   & #5   \\ \hline
Body	 	& \rlgoal   & #6   \\ \hline
Test		& \testgoal & #7   \\ \hline
\end{tabular}
\end{center}}

\def\shortrecipe#1#2#3#4#5{%%
\begin{center}
\begin{tabular}{|l|l|l|}\hline
Phase            &  Goal 		& Activity \\ \hline\hline
\begin{minipage}[t]{.60in}
Contract\\
Purpose and\\
Header\\
\end{minipage} 	& \congoal  & #1 \\ \hline
%Header   	& \hdgoal   & #3   \\ \hline
Examples 	& \exgoal   & #2   \\ \hline
Body	 	& \rlgoal   & #4   \\ \hline
Test		& \testgoal & #5 \\ \hline
\end{tabular}
\end{center}}

\def\asbefore#1{\multicolumn{2}{c|}{see figure~{#1}\raisebox{-10pt}{\rule{0pt}{15pt}}}}

% For dlatex:
\newcommand{\TabSpace}[0]{\makebox[1ex][l]{}} 
\htmlonly 
\newcommand{\TabSpace}[0]{\ } 

\def\beginfig{%
 \begin{rawhtml}
   <hr />
   <blockquote>
   <table bgcolor="beige">
   <tr><td>
 \end{rawhtml}
}
\def\caption#1{\center{Figure: #1}}

\def\endfig{%
 \begin{rawhtml}
   </td></tr>
   </table>
   </blockquote>
   <hr />
 \end{rawhtml}
}

\def\notinhtml{%
 \begin{rawhtml}
  <center>
    <font color=red>The figure is not yet translated into HTML.</font>
  </center>
 \end{rawhtml}
}

\endhtmlonly
\begin{document}
\newpart{Assertions and Contracts}


\section{Types are Your Friends}

A type checker is a theorem prover. As it crawls over your program, it
  confirms your claims about your functions and variables once and for
  all. Example: 


\begin{verbatim}
 ...
 int f(long x) { 
  return ... x ...; 
 }
 ...
\end{verbatim}

When the Java type checker checks the program that contains this code
  fragment, it confirms two of your claims:
\begin{enumerate}
\item when {\tt f} receives a value for {\tt x}, it is an

\item if {\tt f} produces a value, it is an element of the class of
{\tt int} integers.

\end{enumerate}
While these facts don't represent much of the knowledge that went into the
production of the program, it is still more than testing could ever
establish. After all, testing can only confirm a finite number of facts
about {\tt f}.

Knowledge like this comes in handy when you are looking for errors in
  your program. As you inspect your program for potential problems, you do
  not have to re-confirm that {\tt f} maps {\tt long}s to
  {\tt int}s; you may assume it as a given.

{\bf WARNING:} The above is only true for program languages with sound
  type systems. Examples of such languages are Java, ML, and Haskell. In
  contrast, C and C++ do not have sound type systems; a function such as
  {\tt f} can always consume a bit pattern that corresponds to a
  (portion of a string) and produce a (portion of a) bit pattern from a
  structure. Hence, when a C/C++ program breaks, you may not assume
  anything is correct. 


In the ideal world, we should therefore express everything we know about
  our program in the type system. Then, as the type checker blesses the
  program, it would confirm that the program runs correctly for
  ever. Unfortunately, this is impossible in the abstract (if we wish to
  have terminating type checkers) and in the concrete; the type languages
  of programming languages are just not expressive enough. 


Consider this example:

\begin{verbatim}
 prime h(prime x) { 
  return ... x ...
 }
\end{verbatim}
This code fragment specifies that {\tt h} consumes and produces
prime numbers. No current type language supports a {\tt prime} type
at the moment, and including it in a type language would soon run into
problems with the undecidable theory of arithmetic. Thus, even though such
function signatures would be highly useful in, say, the realm of
cryptography, it is impossible to state such facts and to have them
confirmed by the type checker.


\section{Assertions}

{\bf (Types are not friendly enough)}

To overcome this deficiency, we need to generalize the language of types
  to a language of arbitrary assertions. Roughly speaking, an assertion is
  a claim about the values in our computer programs as the (abstract)
  computer executes them. Historically, people have associated stated
  assertions about program variables. Examples: 

\begin{verbatim}
 int p; // p is a prime number
\end{verbatim}

or

\begin{verbatim}
 int x; 
 ...
 x = x * x; // x is now positive 
 ...  
\end{verbatim}

or 

\begin{verbatim}
 Process current; 
 Queue[Process] q; 
 ... 
 q.enq(current); 
 // q is not empty 
 runProcess(q.deq()); 
 // warning: q might be empty now 
 ...
\end{verbatim}

The last two examples also show that assertions are established as the
result of executing a statement, and using assertions, we can argue that
code fragments will execute properly. Due to this role, assertions are
often referred to as pre-conditions and post-conditions (of statements). 


Since English is ambiguous and since assertions (claims of truth) are
  the elements that logic deals with, computer scientists quickly adopted
  the language of logic to express facts. Thus, instead of the above,
  people write: 

\begin{verbatim}
 // forall i such that 2 $\leq$ i < p, 
 // &nbsp; &nbsp; &nbsp; not[(p modulo i) = 0]
 int p; 
\end{verbatim}

and

\begin{verbatim}
 int x; 
 ...
 x = x * x; // x >= 0
 ...  
\end{verbatim}

or 

\begin{verbatim}
 Process current; 
 Queue[Process] q; 
 ... 
 q.enq(current); 
 // not[q.empty()]
 runProcess(q.deq()); 
 // WARNING: q.empty() v not[q.empty()]
 ...
\end{verbatim}


Next people formulated rules for reasoning about program statements,
  especially assignment statements, if statements, and loops. 


\section{Assertions Describe the Behavior of Programs}

Take a look at this program fragment: 
\begin{verbatim}
int x = 0; 
// 1: x = 0 
x = x + 1; 
// 2: x = 1 
\end{verbatim}
The comments are logical assertions about the state of the program. Indeed,
they actually describe the execution of the program. Specifically, the
first assertion describes the entire state after initialization and the
second one the state after the assignment statement. 


For if-statements, we need to take into account a case split: 
\begin{verbatim}
int x = ...; 
int y = ...; 
... 
int max = 0; 
// 1: max = 0 
if (x > y) { 
 // 2a: (x > y) 
 max = x; 
 // 3a: x > y => max = y
 } 
else {
 // 2b: not(x > y)
 // 2b: x $\leq$ y
 max = y; 
 // 3b: x $\leq$ y => max = y
 }
// 4: (x > y => max = x) ^ (x $\leq$ y => max = y) 
\end{verbatim}

In each branch, we can add the condition that succeeded (and calculate with
it: see 2b). Then, at the end of the if-statement, we can combine the two
branches with a logical conjunction. In this example, we may conclude that
no matter what {\tt x} and {\tt y} are, {\tt max} is
equal to the larger of the two.


Last but not least, we can also use assertions to describe the execution
 of loops: 
\begin{verbatim}
int a[];
...
int i = 0; 
int sum = 0; 
// 1: i = 0 ^ sum = 0 
...
// 2: sum = &Sigma; { a[j] | 0 $\leq$ j < i } ^ i = 0
for(i = 1; i < a.length; i++) {
 sum = sum + a[i]; 
 // 3: sum = &Sigma; { a[j] | 0 $\leq$ j < i }
 }
// 4: sum = &Sigma; { a[j] | 0 $\leq$ j < i = a.length }
\end{verbatim}

Here we see that one and the same assertion holds before the loop starts
(2), at the end of each loop body (3), and after the loop stops. The
conclusion is then that {\tt sum} is the sum of all numbers in the
array {\tt a}. Because this assertion holds at all these places, it
is also called a loop invariant (i.e., an assertion that is true across
several statements). 


(Note: the assertion is false at the entry of the loop body. At that
  point {\tt i} was just increased by 1, yet {\tt sum} is
  still the old value.)


For years (1960's through the early 1980's), the holy grail of
  programming language research was the development of a calculus that
  would empower programmers to specify the initial state and the final
  state, and to create the program from these statements. Alternatively,
  the programmers would code something and an automatic theorem proofer
  would verify the result. (See Dijkstra's monograph.)


The effort collapsed when people couldn't figure out how to easily scale
  this kind of reasoning to programming languages with procedure
  definitions and procedure calls (and beyond). The idea of programming
  triples, however, survived. To this day, it is a good idea to describe
  small code fragments as Hoare triples: 

\begin{center}{\tt PRE { programStatement } POST}\end{center}

where {\tt PRE} and {\tt POST} are logical statements
describing the state of the abstract machine. Since people couldn't reason
about several procedures, it is also natural that they started describing
the behavior of procedures with triples. 


\section{Assertions for Method Boundaries}

Although research on program correctness per se failed, the primary use
  of {\tt PRE} and {\tt POST} assertions is to enhance the
  type signature of a method. In other words, if these descriptions are
  added in an interface, they become a powerful tool to specify those
  obligations of callers and callees that go beyond types. Even if such
  assertions no longer describe the entire behavior of a method's body,
  they can still notice basic mistakes that can pollute software for a long
  time. 


Let us look at a canonical example:
\begin{verbatim}
interface Queue {
 // is this queue empty? 
 boolean empty();

 // add Item x to the end of this queue
 Queue enq(Item x); 

 // produce and remove the first Item from the queue
 Item deq(); 

 // how many items are in this queue 
 int size();
}
\end{verbatim}

This Java interface specifies a collection of data structures with four
methods. At first a newly created queue is empty. With {\tt enq}, we can
add an item; with {\tt deq} we can retrieve the item. The descriptions of
the two actions imply that they increase and decrease the {\tt size} of
the queue. Their names and the name of the data structure finally suggest that
{\tt deq} retrieves the item that has been in the queue the longest. In
short, our historical knowledge suggests a first-in/first-out data
structure.

In principle, it is possible to supplement the signatures and purpose
statements of this interface with assertions in an equational logic that
describe the behavior we just described informally. Here are just some sample
equations: 

\begin{verbatim}
$\forall$ X: new Queue().enq(X).deq() = X 
$\forall$ X, Y: new Queue().enq(X).enq(Y).deq() = X  

new Queue().size() = 0 
$\forall$ X: new Queue().enq(X).size() = 1
\end{verbatim}

Given the stateful specification of the queue interface, however, it is
impossible to formulate a simple set of equations or logical assertions that
describe the complete behavior of {\tt Queue}.


Still, many aspects of our informal descriptions can be translated into
logical assertions, specifically into pre- and postconditions. Let's practice
this idea with two questions: 
\begin{enumerate}
\item Can the method be called in all situations? If not, describe when it can be
called.
\item Is there anything special about the result value or the result state of each
method?
\end{enumerate}
Applied to our four methods we get these answers: 
\begin{enumerate}
\item It is always possible to call {\tt empty}. Its result is either true
or false, and there is nothing more specific that we can say now.
\item It is always possible to call {\tt enq}. Its result is a queue that
is not empty. Furthermore, the size after the call is one more than the size
before the call. 

\item It is only possible to call {\tt deq} when the queue contains at
least one item. Otherwise it makes no sense to call it. Furthermore, at the end
of the call the queue contains one item less than before the call. 

\item Finally, it is always possible to call {\tt size}. Its result is
always an integer, though it is also true that this integer is always greater or
equal than 0. 

\end{enumerate}


Here is a rewrite of the same interface with {\tt PRE} and
  {\tt POST} conditions: 
\begin{verbatim}
interface Queue {
 // is this queue empty? 
 boolean empty();
 
 // add Item x to the end of this queue
 Queue enq(Item x); 
 // POST: !(empty()) && size() = OLD[size()] + 1

 // produce and remove the first Item from the queue
 Item deq(); 
 // PRE: !(empty()) 
 // POST: size() = OLD[size()] - 1

 // how many items are in this queue 
 int size();
 // POST: RESULT &ge; 0
}
\end{verbatim}
The conditions contain two special notations. The first is
 {\tt OLD[...]}. It denotes the value of the expression {\em before}
 the method call took place. The second one is {\tt RESULT}. It is the
 value that the method produces. Given these explanations, the pre- and
 postconditions in the revised interface express all those conditions that we
 mentioned above, except for those that provide a full-fledged formal
 description of the methods' behavior. Still, experience shows that such
 additional conditions quickly catch mistakes in the general logic, especially
 in conjunction with other pre- and postconditions. 


Let us look at a second example, the conversions of dollar amounts into
strings and vice versa. Since money amounts require precise calculations, it is
a good practice to develop a separate class of {\tt Amount}s for business
applications. Naturally, in addition to basic arithmetic operations on an
{\tt Amount} we also need an operation that converts an instance to a
{\tt String}:
\begin{verbatim}
interface IAmount {
  // convert the amount into a check-style string 
  // with dollars and cents separated by a dot 
  String print() 
}
\end{verbatim}

Furthermore, we may also want to provide a static factory method that reads a
{\tt String} into an {\tt Amount:}
\begin{verbatim}
class Amount implements IAmount {
  ...
  // convert the given check-style string into an amount
  // assume the string separates the dollar amount from 
  // the cents via a dot 
  public static Amount read(String s) { ... }
  ...
}
\end{verbatim}

In either case, it makes sense to describe the obligation of {\tt print}
or the obligation of {\tt read}'s client with assertions that catch basic
mistakes.


The informal purpose statement implies two major characteristics about the
string so far: 
\begin{enumerate}
\item the string consists of digits except ...
\item ... for the third spot from the right, which is a dot.
\end{enumerate}
Using a forall quantifier, we can express these two constraints as follows: 
\begin{verbatim}
  RESULT[RESULT.length() - 3] = '.' 
  &&
  $\forall$ i in [0..RESULT.length] : i != RESULT.length() - 3 
   => RESULT[i] in ['0'..'9']
\end{verbatim}

As before, the assertion uses both program variables and even programming
notation to express basic ideas but also mathematical symbols, e.g.,
$\forall$. Furthermore, it does not express the complete behavior of either
{\tt print} or {\tt read} but it expresses essential
characteristics. In particular, it clarifies what it means to supply or receive
a check-style string. 


If this condition fails, something is seriously wrong with the method or its
caller. Hence monitoring conditions such as these helps us defend ourselves
against miscommunication among programmers. The key is then to not only add such
statements but to turn them into a part of the running code.


\section{From Assertions to Contracts}

In the business world, a contract is a document that spells out the
obligations of two (or more) participants in a joint activity. For example, a
contract may specify that an oil company delivers oil to a residence and that
the owner in turn pays for the oil. If either of the two parties doesn't live up
to its obligations, the other one will go to a neutral arbiter who decides
whether the claim is correct. If so, we say that one of the two parties got
blamed. 


This scenario suggests that turning assertions into useful program monitoring
tools requires three things: 
\begin{enumerate}
\item the assertions must become executable;
\item a neutral arbiter must monitor the assertions;
\item if an assertion goes wrong, the monitor must assign blame to the party that
fails to live up to its obligations.
\end{enumerate}

In short, assertions convey knowledge and you can be blamed for what you
knowingly violate; converting assertions into contracts does just that. 


We can go about making assertions executable in two ways. On one hand, we can
create a notation for assertions, have programmers express their thoughts in
this notation, and translate it into code. On the other hand, we can simply say
that all boolean expressions of the underlying programming language are
assertions and that the monitor just executes them at opportune moments.

The boolean-expression approach is dominant in industrial languages though
only Eiffel treats these assertions as contracts. Meyer chose to equate boolean
expressions with assertions because of pragmatic reasons, even though he was
keenly aware of the problems. The major problem is of course that a call to a
boolean-value function is an assertion but the execution of this assertion may
have side-effects. That is, it may diverge, raise an exception, or change the
state of a variable. 

Research projects tend to use separate languages for contracts and
assertions. Two prominent examples are the ESC/Java project at the former DEC
SRC lab (short for extended static checking), which uses the Java Modeling
Language (JML), and the SPEC\# project at Microsoft research, which uses its own
notation. The latter provides a programming language, a run-time monitoring
system, and a theorem proofer. If the theorem proofer can verify that a party
always meets its obligations, it can inform the compiler, which will then
disable the corresponding monitoring code. Conversely, if the theorem prover
cannot establish that the party lives up to the contract in all situations, the
compiler will issue appropriate code. 


No matter how contract violations are found, it is important that the guilty
party is blamed and that the error message is a good explanation for the
problem. After all, the purpose of contracts is to protect programmers against
miscommunication and to help them eliminate problems. 


{\bf Note 1:} Unfortunately, most contract systems for Java and Eiffel lack
a semantic foundation. As a result, they may not notice a contract violation at
all, blame the wrong party, or blame the proper party with an incorrect
explanation. Furthermore, the pun between assertion variables and program
variables has prevented programmers from expressing and monitoring assertions
about objects (rather than just plain values).(See Findler's dissertation and
research for detailed explanations.)


{\bf Note 2:} Monitoring contracts can get expensive. It is therefore
important to think about what you want your contract system to check. The goal
is to find the proper compromise between checking basic properties and checking
them without much overhead in time, space, and energy. A basic property of a
method is an assertion whose violation suggests something fundamental went
wrong. For example, if there is no dot at the proper position in the result of
{\tt print} in {\tt Amount}, it is not the desired print
representation of a dollar amount. Checking this one character in the string is
relatively cheap; it adds a constant to the running time. Checking that every
other character is a digit requires a loop. The cost is now linear in the length
of the resulting string but that is acceptable because the routine itself is
probably linear. Adding more than a linear cost would definitely be
questionable.


\section{Sequence Contracts}

In addition to types (signatures) and basic conditions about the
before and after state of the world, we often also want to ensure
that methods are called in the proper order.

Take a look at this simple integer file interface:

\begin{verbatim}
interface IPort {
    // connect port to file
    void open(); 

    // read one integer from file, if the file isn't exhausted
    int read(); 
    // @pre: !out\_of\_intsP()

    // is the file content exhausted? 
    boolean out\_of\_intsP(); 

    // disconnect the port from the file 
    void close();     

    // sequence contract: open . {read | out\_of\_intsP}* . close
}
\end{verbatim}

Presumably, we create an IPort with a name and then work with it: 

\begin{verbatim}
class PortTest {
    public static void main(String argv[]) {
	int sum; 
	int silly[] = {1,2,3}; 
	IPort ip = new PortState(silly);
     
	// ip.open();
	sum = 0; 
	while (!ip.out\_of\_intsP() || true)
	    sum += ip.read(); 
	ip.close(); 
	System.out.println("6 == " + sum); 
    }
}
\end{verbatim}

Clearly, the intention is that consumers call the methods of IPort in a
particular sequence, namely, one call to {\tt open}, followed by calls to
{\tt read} and {\tt out\_of\_intsP}, followed by one (optional) call
to {\tt  close}.

We call this a sequence contract and write something like the
above as

\begin{verbatim}
 open . { read | out\_of\_intsP }* . close
\end{verbatim}

The notation:
\begin{itemize}
\item The braces group things.
\item The brackets make things optional.
\item a . b means a followed by b.
\item a | b means a or b. 
\item a* means 0 or more actions a. 
\item a+ means 1 or more actions a. 
\end{itemize}


An alternative (but equivalent) notation is that of finite state machines:

\begin{verbatim}
  port :

                open()	                          read(), 
       +------------------------------------+     out\_of\_intsP()
       |                                    |   +---------+
       |                                    v   |         |
 +------------+                          +----------+     |
 | ClosedPort |                          | OpenPort |     |
 +------------+                          +----------+     |
       ^                                    |   ^         |
       |                                    |   |         |       
       +------------------------------------+   +---------+
                close()
\end{verbatim}

This notation suggests an implementation via the state pattern [GoF]:


\begin{verbatim}
{\tt 
// Checking sequence contracts via the state pattern. 

// Each class of states is represented with a single instance from a 
// private class. Each state implements the full repertoire of method 
// calls; the Port class itself just forwards each message to the current 
// state. 

class PortState implements IPort {

    // -------------------------------------------------------------------------
    // fields: 

    private int values[]; 

    private IPort state; 

    private OpenPort theOpenPort = new OpenPort(); 

    private ClosedPort theClosedPort = new ClosedPort(); 

    PortState(int values[]) { 
	this.values = values; 
	state = theClosedPort;
    }

    // -------------------------------------------------------------------------
    // the public interface 

    public void open() { 
	state.open(); 
    }

    public int read() {
	return state.read();
    }

    public boolean out\_of\_intsP() {
	return state.out\_of\_intsP(); 
    }

    public void close() {
	state.close();
    }	

    // -------------------------------------------------------------------------
    // the private classes 

    private class OpenPort implements IPort {

	private int ptr = 0; 

	public void open() { 
	    System.out.println("can't call open on open file"); 
	    System.exit(-1); 
	}

	public int read() {
	    preRead();          
	    int r = values[ptr]; 
	    ptr++; 
	    return r; 
	}

	public boolean out\_of\_intsP() {
	    return (ptr >= values.length); 
	}

	public void close() {
	    state = theClosedPort; 
	}	

	// checking the precondition for read
	private boolean preRead() { 
	    if (ptr < values.length)
		return true; 
	    System.out.println("precondition violation"); 
	    System.exit(-1); 
	    return false; 
	}
    }

    private class ClosedPort implements IPort {

	public void open() { 
	    state = theOpenPort; 
	}

	public int read() {
	    Ouch(); return -1; 
	}

	public boolean out\_of\_intsP() {
	    Ouch(); return false; 
	}

	public void close() {
	    Ouch(); 
	}	

	private void Ouch() {
	    System.out.println("sequence contract violation");
	    System.exit(-1); 
	}
    }
}
}
\end{verbatim}


\section{Contract Violations and Fault-Tolerant Programming}

When contracts go wrong, the system is in trouble. Contracts monitor basic
properties that should never go wrong. At the same time, however, the purpose of
contracts is to check on the communication between disjoint parts of the
system, and in common cases, the various parts of a system are not equals. If a
lesser part fails, the overall system might just work at acceptable levels
without the failing part. 

This reasoning suggests a natural combination of contracts with techniques
from fault-tolerant systems. Specifically, imagine that the composition of the
system is flexible. While it runs, various components can join or leave the
system. Naturally, the system contains essential components for which this is
not true, but it might be true for peripheral or for redundant parts. Then, when
one of those redundant or peripheral subsystems fails, the kernel of the system
can "unplug" the failing part and inform the rest of the components to work
without it. Indeed, depending on how the system is configured, the remaining
components don't ever need to know. 


Consider a system that is playing the games. The game administrator is
obviously an essential component. If it fails, nothing else can replace it. In
contrast, if one of the player fails---perhaps due to a bad connection or
perhaps due to cheating---the game system can obviously do without the player.

The internet is another example of a fault-tolerant system. When computers
notice that some other computer no longer responds with acknowledgments, they
route network traffic around the failing computer. If the failing computer is a
mail server, they may hold the mail for several days and attempt to redeliver it
on a regular basis.

In general, the use of assertions (invariants) and contracts has led to a
number of recent attempts to produce self-healing or self-repairing
systems. While some ideas have already appeared in industrial contexts, for many
it is too early to tell whether they will make it out of the research lab.

\section{Bibliography}

\begin{itemize}
\item Dijkstra. A Discipline of Programming. Prentice Hall. 1974.

\item Findler. Behavioral Software Contracts. Dissertation, Rice University,
2001. 

\item Flanagan, Leino, Lillibridge, Nelson, Saxe, and Stata.
 Extended static checking for Java. PLDI 2002.


\item Meyer. Applying "Design by Contract". IEEE Computer, Oct 1992.

\item 
Barnett, Leino, Schulte. 
The Spec\# programming system: An overview. CASSIS 2004.
 
\end{itemize}

\end{document}
