\documentclass{article}

\usepackage{times}

\def\email#1{ {\tt <#1>} }

\def\dollars#1{\$#1}
\def\dollars#1{\relax}

\def\specialtopic#1{\subsection*{#1}} % don't start table!

\def\topic#1{\subsection*{#1}}
\def\subtopic#1{\bigskip\noindent{#1}}
\def\endtopic{\vskip1.cm}

\def\litem#1 {\par\begingroup\clubpenalty=10000%
           \def\par{\endgraf\endgroup}%
	   \noindent\hangindent=2cm\hangafter=0%
	   \hbox to0pt{\hskip-2cm#1\hfill}}

% ----------------------------------------------------------------------------------

\begin{document}

\begin{center}
{\bf Matthias Felleisen}
\ \\
{College of Computer Science}\\
{Northeastern University}\\
{Boston, Massachusetts 02115}\\
\verb|matthias@ccs.neu.edu|\\
\verb|http://www.ccs.neu.edu/home/matthias/|\\
\end{center}

\topic{Education} 
\litem{Aug. 1987} Ph.D., Indiana University, Bloomington, Indiana
\litem{Dec. 1983} Dipl. Wirtschafts-Ingenieur, Universit\"at Karlsruhe, Germany
\litem{Aug. 1981} M.S., The University of Arizona, Tucson, Arizona
\endtopic

\topic{Professional Experience} 
\litem{2001--pr.} Trustee Professor, Northeastern University, Boston, Massachusetts
\litem{1993--2001} Professor, Rice University, Houston, Texas
\litem{1992--1993} Associate Professor, Rice University, Houston, Texas
\litem{1987--1992} Assistant Professor, Rice University, Houston, Texas
\litem{1984--1986} Research/Teaching Assistant, Indiana University, Bloomington
\litem{summer '85} Research Associate, MCC, Austin, Texas
\litem{1982--1983} Programmer, IDS, Karlsruhe, Germany
\litem{1980} Associate Instructor, Universit\"at Karlsruhe, Germany
\endtopic

\topic{Extended Visits} 
\litem{June 1996} Ecole Normale Superieure, Paris, France
\litem{1993--1994} Carnegie Mellon University, Pittsburgh, PA 
\litem{May 1991} Ecole Normale Superieure, Paris, France
\endtopic

\topic{Research and Teaching Interests} 

I conduct research on all aspects of program design and programming
language design. I occasionally investigate the role of language technology
in software engineering.

As for teaching, I am interested in all courses in the computer science
core curriculum, covering program design for applications, components, and
systems plus the underlying logic.\footnote{The study of logics relates to
programming like analysis relates to engineering.}

My teaching interests extend to the use of programming in K-12 education,
especially in mathematics and the sciences. 
\endtopic

\topic{Major Products} 

I am the founder of PLT, a distributed research group that jointly
produces the {\bf Racket} programming language, the {\bf DrRacket} IDE, and
plug-in tools (the Stepper, the Macro Debugger, the PLT Web Server, etc).
The tools are used in educational outreach projects as well as industrial
projects.
\endtopic

\topic{Awards/Fellowships} 
\litem{2012} ACM SIGPLAN Lifetime Achievement Award 
\litem{2012} ACM SIGPLAN Most Influential ICFP Paper Award 
\litem{2011} ACM SIGCSE Outstanding Educator Award 
\litem{2009} ACM Karl V. Karlstrom Outstanding Educator Award 
\litem{2007} ACM Fellow 
\litem{1998} George R. Brown Award for Innovative Teaching
\litem{1986--1987} IBM Graduate Research Fellowship
\litem{1982--1983} Konrad-Adenauer Stipendium
\litem{1980--1981} Fulbright Fellowship
\litem{1979} Preis des Landes Baden-W\"urttemberg for Best {\it Vordiplom\/} of the Year
\endtopic

\topic{Active Grants} 
\litem{2014--2017} Compiler Coaching. With Tobin-Hochstadt (Indiana). NSF. \dollars{500,000}
\litem{2014--2017} Run Your Research with Redex. With Flatt (Utah) and Findler (Northwestern). NSF. \dollars{800,000}
\litem{2011--2014} Semantics Engineering for Scripting Languages. With Krishnamurthi (Brown) and Findler
(Northwestern). NSF. \dollars{1,299,000} 
\litem{2010--2015} Gnosys. With O. Shivers, M. Wand (Northeastern). Darpa. \dollars{2,400,000}
\litem{2009--2015} A Hot-House for Programming Languages. NSF. \dollars{684,000}

\subtopic{Past Grants} 
\litem{2009--2012} Software Contracts in a Higher-Order World. With R. Pucella. AFOSR. \dollars{480,000}
\litem{2009--2013} Integrating Mechanized Logic into the SE Curriculum. With R. Page (OU). NSF. \dollars{500,000}
\litem{2006--2009} Interface-Oriented Programming. NSF. \dollars{300,000}
\litem{2006--2009} Using Market Forces to Improve the Design of Software. With R. Findler (U. Chicago). NSF. \dollars{104,000} 
\litem{2005--2009} Support for  Language-Oriented Programming in PLT Scheme. NSF. \dollars{200,000}
\litem{2004--2008} Well-founded Behavioral Software Contracts. With R. Findler (U. Chicago). NSF. \dollars{180,000}
\litem{2002--2008} Scheme on .NET. With W. Clinger, M. Wand. Microsoft Research. \${950,000}
\litem{2003--2006} Robust Web Services. With S. Krishnamurthi (Brown). NSF. \dollars{300,000}
\litem{2001--2005} Computing Education for Every Student in Secondary Schools. NSF. \dollars{1,899,995} \\
Supplemental grants: Cord.org, \${30,000}; Exxon, \${50,000}.
\litem{2000--2005} Integrating Logic in the Computer Science Curriculum. With M. Vardi (Rice). NSF. \dollars{772,910}
\litem{2000--2002} NextGen: A Programming Environment for Generic Java. With R. Cartwright (Rice). Texas Advanced Technology Program. \dollars{158,000$^\dagger$}
\litem{1997--2001} Modular Program Analyses for Higher-Order Programming Languages. NSF. \dollars{233.000.}  
\litem{1997--2001} Re-inventing Computing Education in High Schools. With R. Cartwright (Rice). US DoEd. \dollars{303,000}  
\litem{1997--2001} Re-inventing Computing Education in High Schools. Exxon. \dollars{43,000$^\dagger$}  
\litem{1997--2001} Exploring a ``Safe'' Approach to Software Engineering. With R. Cartwright (Rice). NSF. \dollars{895,000$^\ddagger$}  
\litem{1997--2000} A Smart Programming Environment for Java. With R. Cartwright (Rice). NSF. \dollars{151,000.}  
\litem{1998--2000} A Static Debugger for Java. With R. Cartwright (Rice). Texas Advanced Technology Program. \dollars{133,000$^\dagger$} 
\litem{1996--1998} Smart Programming Environments. With R. Cartwright (Rice). NSF. \dollars{164,000.} 
\litem{1996--1998} Diagrammatic Reasoning in Hardware Verification. With M. Vardi (Rice). NSF. \dollars{48,000.} 
\litem{1996--1997} From Scheme to Object-Oriented Programming. NSF. \dollars{44,000$^\ddagger$}  
\litem{1994--1997} Can We Unify the Programming Curriculum? With R. Cartwright (Rice). NSF. \dollars{944,000. $\ddagger$}  
\litem{1992--1995} Fully Abstract Semantics for Practical Programming Languages. With R. Cartwright (Rice). NSF. \dollars{480,000.} 
\litem{1992--1995} Felleisen, M. Soft Typing. Army. \dollars{90,000.} 
\litem{1991--1994} Practical Softtyping. With R. Cartwright (Rice). Texas Advanced Technology Program. \dollars{198,000.} 
\litem{1990--1992} On the Expressive Power of Programming Languages. NSF.  \dollars{114,000.} 
\litem{1988--1990} The Semantic Foundation of Program Optimization. With R. Cartwright (Rice). NSF. \dollars{48,000.} 
\litem{1987--1990} An Automated System for Deriving Efficient Parallel Programs. With K. Kennedy, R. Cartwright, K.~Cooper (Rice).
 DARPA. \dollars{1,291,000.} 
\endtopic

\dollars{ \noindent Remarks: ``joint'' refers to joint PD/PI, ``with''
means Co-PD/PI for NSF proposals; $\ddagger$ EI grants include Rice
matching dollars (typically 30-40\% of total); $\dagger$: these amounts are
free of overhead.} 

\topic{Professional Activities} 

\litem{2009-pr.} Editor in Chief: {\it Journal of Functional Programming\/} \\

\litem{2004-2008} Editor: {\it Journal of Functional Programming\/} \\

\litem{2001-2008} Editor: {\it Journal of Functional Programming\/} \\
in charge of the Educational Pearls section;

\begin{enumerate}
\item[] ``Welcome'' (2003, issue 4);
\item[] ``{The Structure and Interpretation of the Computer Science
Curriculum}'', with Findler, Flatt, and Krishnamurthi, Vol~14(4), 2004, 365--378
%\item[] ``Continuations and the Web'', in preparation. 
\end{enumerate}

\litem{1996-2001.} Editorial Board: {\it Journal of Functional Programming\/}

\litem{2012} Program Committee Chairman, {\it The 2013 European Symposium on Programming\/}.
\litem{2006} Program Committee Chairman, {\it The 2007 ACM Symposium on the Principles of Programming Languages\/}.
\litem{2006} Program Committee, {\it The 2006 CUFP Workshop\/} (at {\it ICFP 2006\/}).
\litem{2005} Program Committee,  {\it Object-Oriented Programming, Systems, Languages, and Applications 2005\/}.
\litem{2004} Program Committee,  {\it International Conference on Functional Programming\/}. 
\litem{2004} Panelist. National Academy of Science, CSTB: {\it Building Dependable Systems\/}.  
\litem{2003} Host, {\it Scheme and Functional Programming}. 
\litem{2003} Program Committee,  {\it Continuation Workshop '04\/} 
\litem{2003} Selection Committee,  {\it HOSC---Issue on the Scheme workshops.\/} 
\litem{2000} Organizer, {\it Scheme and Functional Programming}.
\litem{2000} Program Committee, {\it Practical Applications of Declarative Programming Languages, \bf2000}.
\litem{1999} Program Committee, {\it European Symposium on Programming \bf2000}.
\litem{1999} Co-Organizer, {\it Functional and Declarative Languages in Education, \bf1999}.
\litem{1997--2000} Steering Committee. {\it International Conference on Functional Programming\/} 
\litem{1997--1999} Professional Activities Committee, SIGPLAN
\litem{1998} General Chair, {\it International Conference on Functional Programming\/}.
\litem{1997} Program Committee, {\it Conference on Declarative Programming Languages in Education\rm
             (at {\it Programming Languages, Implementations, and Logics\/})\/}.
\litem{1996} Organizer,  {\it Workshop on Functional Languages in the Introductory Curriculum} 
\litem{1996} Co-Organizer,  {\it Scheme Implementors' Workshop\/} 
\litem{1995} Program Committee,  {\it International Conference on Functional Programming\/}.
\litem{1994} Program Committee,  {\it Conference on Lisp and Functional Programming\/}  (LFP '94).
\litem{1993} Program Committee,  {\it Workshop on State in Programming Languages\/}.
\litem{1993} Selection Committee,  {\it LASC---Special Issue on Continuations.\/} 
\litem{1993} Program Committee,  {\it Conference on Functional Programming Languages and Architecture.\/}  
\litem{1991} Program Committee,  {\it Continuation Workshop '92\/} 
\litem{1989} Program Committee,  {\it 17th ACM Symposium on the Principles of Programming Languages\/}.
\endtopic

% \specialtopic{Current Long-Term Projects} 


% \specialtopic{On to Component Programming in High-Level Languages}

% My intention for the next five years is to conduct research on
% next-generation component systems. Our experience with DrScheme suggests
% that building systems from components requires high-level languages with
% fine-grained operating systems capabilities. System builders must have
% tools for controlling the resource consumption of their
% systems. Furthermore, they must have linguistic constructs that concisely
% express the architecture of the system and that permit programmers to
% specify some crucial properties of components.  Using this architecture
% description, we can conduct system analyses that validate guarantees as
% much as possible and insert monitoring code where necessary. The research
% will draw on results from programming languages (design, implementation,
% static analysis) operating systems, and software engineering.

% \endtopic

\topic{Outreach \& Software Projects} 

\litem{1995--pr.} Program by Design (formerly known as TeachScheme!),
includes Bootstrap for middle school students

\litem{1994--pr.} Racket. (formerly known as PLT Scheme) Collaborators:
Flatt (Utah), Findler (Chicago), plus close to 100 committers to the code
base. 

\litem{1985--1989} Prolog-in-Scheme and Schelog: macro-embeddings of
Prolog into Scheme. Main collaborator: D. Sitaram.

\litem{1985--1989} extend-syntax: a pattern-oriented, hygienic macro
system for Scheme88. Main collaborator: J. Greiner.
\endtopic

\topic{Books} 

\litem{2013} {\it Realm of Racket: For Freshmen, by Freshmen}. NoStarch Press.
(With 
Bryce, DeMaio, Florence, 
Lin, Lindemann, Nussbaum, 
Peterson, Plessner, Van Horn, 
Barksi)

\litem{2010} {\it Semantics Engineering with PLT Redex}. MIT Press. 
Korean translation: 2012.
(With R. Findler, M. latt). 

\litem{2001}  {\it How to Design Programs\/}. MIT Press. 
Korean translation: 2012.
Spanish translation: 2008.
Chinese translation: 2003. 
Polish translation: 2002.
(With R. Findler, M. Flatt, S. Krishnamurthi)

\litem{1998}  {\it A Little Java, A Few Patterns\/}, MIT Press. --- Japanese
translation appeared in 1998. (With D.\ P.\ Friedman) 

\litem{1998}  {\it The Little MLer\/}. MIT Press. (With D.P. Friedman)

\litem{1996}  {\it The Seasoned Schemer}. MIT Press. (With D.P. Friedman)

\litem{1996}  {\it The Little Schemer\/}  (Fourth Edition).  MIT Press. --- 
Third Edition: 1989, MacMillan (SRA division). 
Trade Edition: 1987, MIT Press.  
Second Edition: 1985, Science Research Associates.
Japanese translation: 1990, McGraw Hill International. 
French translation: 1991, Mason Publishers. 
(With D.P. Friedman)
\endtopic

\topic{Refereed Publications: Journals and Book Chapters} 
% -------------------------------------------------------------------
\litem{2013} {Strickland, T.S., Dimoulas, C., Takikawa, A., M. Felleisen}. 
 Contracts for First-Class Classes
 {\it ACM Transactions on Programming Languages and Systems}, 
 35(3), 11:1--58
% -----------------------------------------------------------------------------
\litem{2011} {Dimoulas, C., M. Felleisen}. 
 On Contract Satisfaction in a Higher-Order World.
 {\it ACM Transactions on Programming Languages and Systems}, 
 33(5), 16:1--16:29.
% -------------------------------------------------------------------
\litem{2010} {Culpepper, R., M. Felleisen}.
 Debugging Macros. 
 {\it Science of Computer Programming \bf 75}(7), 496--515.
% -------------------------------------------------------------------
\litem{2009} {Felleisen, M., S. Krishnamurthi}.
 Viewpoint: Why computer science doesn't matter. 
 {\it Communications of the ACM \bf52\rm(7)}, 37--40.
% -------------------------------------------------------------------
\litem{2007} {Krishnamurthi S.,  P. Hopkins,  J. McCarthy, P. Graunke, G. Pettyjohn , M. Felleisen}.
 Implementation and Use of the PLT Scheme Web Server. {\it Journal of
 Higher-Order and Symbolic Computing \bf20\rm(4)}, 431--460.
% -------------------------------------------------------------------
\litem{2006} {Krishnamurthi S., R. B. Findler, P. Graunke, and Matthias
 Felleisen}.  Modeling Web Interactions and Errors.  In Dina Goldin, Scott
 Smolka, Peter Wegner (eds.), {\it Interactive Computation: The New
 Paradigm\/}, Spring\-er Verlag, 2006, 255--276.
% -------------------------------------------------------------------
\litem{2004} {Matthews, J., R. Findler, P. Graunke, S. Krishnamurthi, M. Felleisen}. 
 Automatically Restructuring Programs for the Web. 
 {\it Journal of Automated Software Engineering.\/}
 Special issue on {\it ASE\/} 2002. Vol.~11(4), October 2004, 337--364.

% -------------------------------------------------------------------
\litem{2004} {Clements, J. and M. Felleisen}.
 A tail-recursive machine semantics for stack inspection. 
 {\it ACM Transactions on Programming Languages and Applications}.
 November 2004, 1029--1052. 

% -------------------------------------------------------------------
\litem{2004} {Felleisen, M., R. Findler, M. Flatt, S. Krishnamurthi}. 
 The TeachScheme!\ Project: Computing and Programming for Every Student
 {\it Journal of Computer Science Education\/}
 Special issue on K12 education, Vol.~14(1), 2004, 55--77.
% -------------------------------------------------------------------
\litem{2002} {Findler, R., M. Flatt,  S. Krishnamurthi, P. Steckler, and M. Felleisen}. 
The DrScheme Programming Environment. 
{\it Journal of Functional Programming\/ \bf12}(2), 159--182.
% -------------------------------------------------------------------
\litem{1999} {Flanagan, C., M. Felleisen}.
Compositional set-based analysis. 
{\it ACM Transactions on Programming Languages and Applications \bf21} (2), 
1999, 369--415. 
% -------------------------------------------------------------------
\litem{1999} {Flanagan, C. and M. Felleisen}.
 The semantics of futures and an application. 
 {\it Journal of Functional Programming \bf9} (1), 1999, 1--31.
% -------------------------------------------------------------------
\litem{1998}  {Flatt, M., S. Krishnamurthi, and M. Felleisen}.
 A programmer's reduction semantics for classes and mixins. 
 J. Alves-Foss (Ed.), {\it Formal Methods for Java\/}.
 Lecture Notes in Computer Science {\bf 1523}.
 Springer Verlag, Berlin, 1998, 241--270.
% -------------------------------------------------------------------
\litem{1997} Ariola, Z. and M. Felleisen, 
The Call-By-Need Lambda Calculus.
{\it Journal of Functional Programming \bf7\/} (3), 1997, 265--301.
% -------------------------------------------------------------------
\litem{1996} Felleisen, M. and S. Weeks.
On the orthogonality of assignments and procedures in algol.
In {\it Algol-like  Languages}, P. O'Hearn and R.D. Tennent, Eds.
{Birkh{\"a} user}, 1996, 101--123.
% -------------------------------------------------------------------
\litem{1996}  {Cartwright, R. and M.\ Felleisen}. 
Program Verification through Soft Typing. 
{\it ACM Computing Surveys\/}, June 1996, 349--351.
% -------------------------------------------------------------------
\litem{1994} Wright, A. and M. Felleisen.
A syntactic approach to type soundness.  Department of Computer
Science, Rice University, Technical Report No 160.  {\it Information and
Computation \bf115} (1), 38--94.
% -------------------------------------------------------------------
\litem{1994} {Cartwright, R.S., Curien, P.-L., and M. Felleisen}.
Fully abstract models of observably sequential languages.
{\it Information and Computation \bf111} (2), 1994, 297-401.
% -------------------------------------------------------------------
\litem{1993}  {Sabry, A. and M. Felleisen}.
Reasoning with programs in continuation-passing style.
{\it Lisp and Symbolic Computation \bf6} (3/4), 289--360.
% -------------------------------------------------------------------
\litem{1992} Felleisen, M. and R. Hieb. 
The revised report on the syntactic theories for control and state.
{\it Theoretical Computer Science \bf102},~235--272.
% -------------------------------------------------------------------
\litem{1991} Felleisen, M. 
On the expressive power of programming languages.
{\it Science of Computer Programming~\bf17}  (Special issue on ESOP
papers),~35--75. 
% -------------------------------------------------------------------
\litem{1990} Sitaram, D. and M. Felleisen.
Control delimiters and their hierarchies.
{\it Lisp and Symbolic Computation \bf3} (1), 67--100.
% -------------------------------------------------------------------
\litem{1989} Felleisen, M. and D.P. Friedman.
A syntactic theory of sequential state.
{\it Theoretical Computer Science \bf69} (3), 243--287.
% -------------------------------------------------------------------
\litem{1987} Felleisen, M.
Reflections on Landin's J-operator: A partly historical note.
{\it Journal of Computer Languages \rm(Pergamon Press) \bf12} (3/4),
197--207. 
% -------------------------------------------------------------------
\litem{1987} Felleisen, M., D.P. Friedman, E. Kohlbecker, and B. Duba.
A syntactic theory of sequential control.
{\it Theoretical Computer Science \bf52} (3), 205--237.
% -------------------------------------------------------------------
\litem{1986}  {Felleisen, M. and D.P. Friedman}. 
Control operators, the SECD-machine, and the $\lambda$-calculus. 
In {\it Formal Description of Programming Concepts III\/}, edited by 
M. Wirsing. Elsevier Science Publishers B.V. (North-Holland), Amsterdam, 1986,
193--217.
% -------------------------------------------------------------------
\litem{1986} Felleisen, M. and D.P. Friedman.
A closer look at export and import statements.
{\it Journal of Computer Languages \rm(Pergamon Press) \bf11} (1), 28--37.  
% ------------------------------------------------------------------- 
\endtopic

\topic{Conference Publications}  

\litem{2017} {Tobin-Hochstadt, S., M. Felleisen, et al}.
 Migratory typing, ten years later. 
 In {\it SNAPL: Second Summit On Advances In Programming Languages\/}, 
 to appear. 

\litem{2016} {Dimoulas, C., M New, R.B. Findler, M. Felleisen}.
 Oh Lord, don't let contracts be misunderstood. 
 In {\it Proceedings International Conference on Functional Programming\/}, 2016, 456--468.

\litem{2016} {Garnock-Jones, T.and M. Felleisen}.
 Coordinated Concurrent Programming in Syndicate.
 In {\it Proceedings 2016 European Symposium on Programming\/}, 310--336.

\litem{2016} {Takikawa, A., D. Feltey, B. Greenman, M. New, J. Vitek, M. Felleisen}.
 Is sound gradual typing dead? 
 In {\it Proceedings 42nd ACM Symposium on Principles of Programming Languages\/}, 2016, 456--468.

\litem{2015} {Felleisen, M.,
R.B. Findler, 
M. Flatt,
S. Krishnamurthi,
E. Barzilay,
J. McCarthy, 
S. Tobin-Hochstadt}.
 The Racket manifesto. 
 In {\it SNAPL: The Inaugural Summit On Advances In Programming Languages\/}, 
 113--128. 

\litem{2015} {Takikawa, A., D. Feltey, S. Tobin-Hochstadt, R.B. Findler, M. Flatt, and M. Felleisen}.
 Towards Practical Gradual Typing. 
 In {\it Proceedings 2015 European Conference on Object-Oriented
 Programming\/}, 4--27.

\litem{2015} {St-Amour, V., L. Andersen, and M. Felleisen}.
 Feature-specific Profiling 
 In {\it Proceedings 2015 European Symposium on Compiler Construction\/},
 49--68. 

\litem{2014} {Garnock-Jones, T., S. Tobin-Hochstadt, and M. Felleisen}.
 The Network as a Language Construct.
 In {\it Proceedings 2014 European Symposium on Programming\/}, 349--360.

\litem{2014} {Chang, S., M. Felleisen}.
 Profiling for Laziness.
 In {\it Proceedings 41st ACM Symposium on Principles of Programming Languages\/}, 2014, 349--360.

\litem{2012} {V. St-Amour, S. Tobin-Hochstadt, and M. Felleisen}.
 Optimization Coaching.
 In {\it Proceedings 2012 Symposium on Object-oriented Programming Systems, Languages, and Applications}, 163--178. 

\litem{2012} {A. Takikawa, S. Strickland, C. Dimoulas, S. Tobin-Hochstadt, and M. Felleisen}.
 Gradual Typing for First-Class Classes.
 In {\it Proceedings 2012 Symposium on Object-oriented Programming Systems, Languages, and Applications}, 793--810. 

\litem{2012} {Chang, S. and M. Felleisen}.
 The Call-by-need Lambda Calculus, Revisited. 
 In {\it Proceedings 2012 European Symposium on Programming\/}, 128--147.

\litem{2012} {Dimoulas, C., Tobin-Hochstadt, S., and M. Felleisen}.
 Complete Monitors for Behavioral Contracts.
 In {\it Proceedings 2012 European Symposium on Programming\/}, 214--233.

\litem{2012} {Klein, C., J. Clements, C. Dimoulas, C. Eastlund,
 M. Felleisen, M. Flatt, J. McCarthy, J. Rafkind, S. Tobin-Hochstadt, R. B. Findler}.
 Run Your Research. 
 In {\it Proceedings 39th ACM Symposium on Principles of Programming Languages\/}, 2012, 285--296.
 
\litem{2012} {St-Amour, V., S. Tobin-Hochstadt, M. Felleisen}.
 Typing the Numeric Tower. 
 In {\it Proceedings 2012 Conf. Principles and Practice of Declarative Programming}, 289--304.

\litem{2011} {Chang S., E. Barzilay, J. Clements, and M. Felleisen}.
 From Stack Traces to Lazy Rewriting Sequences. 
 In {\it Proceedings 23rd Symposium on Implementation and Application of
 Functional Languages}, 100-0115.

\litem{2011} {Tobin-Hochstadt, S., V. St-Amour, R. Culpepper, M. Flatt, M. Felleisen}.
 Languages as Libraries.
 In {\it Proceedings Symposium on Programming Languages: Design and
 Implementation\/}, 132--141. 

\litem{2011} {Dimoulas, C., R. Findler, C. Flanagan, M. Felleisen}. 
 Correct Blame for Contracts: No More Scapegoating.
 In {\it Proceedings 38th ACM Symposium on Principles of Programming Languages\/}, 2011, 215--226.

\litem{2010} {Strickland, T.S., M. Felleisen}.
 Contracts for first-class modules.
 In {\it Proceedings Sixth Symposium on Dynamic Languages 2010}. 
 		  
\litem{2010} {Eastlund C., M. Felleisen}
 Hygienic macros for ACL2. 
 In {\it Proceedings 2010 Symposium on Trends in Functional Programming}, 84--101.
 		  
\litem{2010} {Chang, S., D. van Horn, M. Felleisen}
 Evaluating Call-By-Need on the Control Stack. 
 In {\it Proceedings 2010 Symposium on Trends in Functional Programming}, 1--15. 

\litem{2010} {Culpepper, R. and M. Felleisen}.
 Fortifying macros. 
 In {\it Proceedings 2010 ACM International Conference on Functional Programming\/}, 2010, 235--246.

\litem{2010} {Strickland, T.S., M. Felleisen}.
 Implementing General Contract Boundaries. 
 In {\it Proceedings 21st Symposium on Implementation and Application of Functional Languages}. 

\litem{2009} {Strickland, T.S., M. Felleisen}.
 Contracts for First-Class Modules.
 In {\it Proceedings Fifth Symposium on Dynamic Languages 2009}, 27--38.

\litem{2009} {Eastlund C., M. Felleisen}
 Making induction manifest in modular ACL2. 
 In {\it Proceedings 2009 Conf. Principles and Practice of Declarative Programming}, 105--116.

\litem{2009} {Dimoulas C., R. Pucella, M. Felleisen}.
 Future contracts. 
 In {\it Proceedings 2009 Conf. Principles and Practice of Declarative Programming}, 195--206.

\litem{2009} {Felleisen, M.,R.B. Findler, M. Flatt, S. Krishnamurthi}.
 A functional I/O system or, fun for freshman kids. 
 In {\it Proceedings 2009 ACM International Conference on Functional Programming\/}, 2009, 47--58 

\litem{2009} {Strickland, S., Tobin-Hochstadt, S., and M. Felleisen}.
 Practical Variable-Arity Polymorphism. 
 In {\it Proceedings 2009 European Symposium on Programming\/}, 32--46.

\litem{2009} {Eastlund, C. and M. Felleisen}.
 Toward a Practical Module System for ACL2. 
 In {\it Proceedings 9th ACM Symposium on Practical Aspects of Declarative Languages\/}, 2009, 46--60. 

\litem{2008} {Tobin-Hochstadt, S. and M. Felleisen}.
 The Design and Implementation of Typed Scheme. 
 In {\it Proceedings 35th ACM Symposium on Principles of Programming
 Languages\/}, 2008, 395--407.

\litem{2007} {Flatt, M., R.B. Findler, G. Yu, M. Felleisen}.
 Adding delimited and composable control to a production programming environment.
 In {\it Proceedings 2007 ACM International Conference on Functional
 Programming\/}, 2007, 165--177.

\litem{2007} {Culpepper. R and M. Felleisen}.
 Debugging Macros, 
 In {\it Proceedings Generative Programming and Component Engineering\/}, 2007, 135--144.

\litem{2006} {Flatt, M., R.B. Findler, M. Felleisen}.
 Scheme with Classes, Mixins, and Traits. 
 In {\it Proceedings Asian Symposium on Programming Languages and Systems (ASPLA
 2006)\/}, 270--289.

\litem{2006} {Tobin-Hochstadt, S., M. Felleisen}.
 Interlanguage Migration: From Scripts to Programs.
 {\it Proceedings Dynamic Languages Symposium (Symposium on Object-oriented Programming Systems, Languages, and Applications 2007 Track)\/}, 964--974.

\litem{2006} {Meunier, P., R. Findler, and M. Felleisen}
 Modular Set-Based Analysis from Contracts.
 In {\it Proceedings 33rd ACM Symposium on Principles of Programming
 Languages\/}, 2006, 218--231. 

\litem{2005} {Pettyjohn, G., J. Clements, J. Marshall, S. Krishnamurthi,
 M. Felleisen} 
 Continuations from Generalized Stack Inspection. 
 In {\it Proceedings 2005 ACM International Conference on Functional
 Programming\/}, 216--227.

\litem{2005} {Cobbe, R. and M. Felleisen}
 Environmental acquisition revisited.
 In {\it Proceedings 32nd ACM Symposium on Principles of Programming
 Languages\/}, 2005, 14--25.

\litem{2004} {Culpepper, R., M. Felleisen}
Taming macros. 
In {\it Proceedings 2004 GPCE Symposium\/}, 225--243.

\litem{2004} {Antoniu, T., P. Steckler,  S. Krishnamurthi, Erich Neuwirth,
M. Felleisen} 
Validating the unit correctness of spreadsheet programs. 
In {\it Proceedings 2004 International Conference of Software Engineering\/}, 439--448.

\litem{2004} {Matthews, J., R. Findler, M. Flatt, M. Felleisen}.
A Visual Environment for Developing Context-Sensitive Term Rewriting Systems. 
In {\it Proceedings 2004 International Conference on Rewriting Techniques and
Applications\/}, 301--311.

\litem{2004} {Findler, R., M. Flatt, M. Felleisen}.
Semantic casts.
In {\it Proceedings 2004 European Conference of Object-Oriented Programming\/},
364--389. 

\litem{2003} {Graunke, P., R. Findler, S. Krishnamurthi, M. Felleisen}. 
Modeling Web Interactions. 
In {\it Proceedings 2003 European Symposium on Programming\/}, 238--252.

\litem{2003} {Clements, J., M. Felleisen}. 
A tail-recursive semantics for stack-inspection.
In {\it Proceedings 2003 European Symposium on Programming\/}, 22--37.

\litem{2002} {Felleisen, M.}.
Developing Interactive Web Programs
In {\it Proceedings 2002 Oxford Summer School on Advanced Functional
Programming\/}, Springer Lecture Notes, 100--128.

\litem{2002} {Findler, R., M. Felleisen}.
Contracts for higher-order functions.
In {\it Proceedings 2002 International Conference on Functional Programming\/},
 48--59. 

\litem{2002} {Logan, M., M. Felleisen, D. Blank-Edelman}. 
Environmental acquisition in network management. 
In {\it Usenix LISA\/}, 175--184. 

% ------------------------------------------------------------------- 
\litem{2001} {Findler, R., M. Latendresse, M. Felleisen}.
Behavioral contracts and behavioral subtyping.
In {\it Proceedings 2001 SIGSOFT Conference on the Foundations of Software
Engineering.\/} 229--236.
% ------------------------------------------------------------------- 

\litem{2001} {Findler, R., M. Felleisen}.
Contract soundness for object-oriented languages. 
In {\it Proceedings 2001 Symposium on Object-oriented Programming Systems, Languages, and Applications}, 1--15.
% ------------------------------------------------------------------- 
\litem{2001} {Graunke, P., R. Findler, S. Krishnamurthi, M. Felleisen}.
Automatically restructuring programs for the web.
In {\it Automated Software Engineering\/}, 211--222.
% ------------------------------------------------------------------- 
\litem{2001}  {Graunke, P., S. Krishnamurthi, S. Van Der Hoeven, M. Felleisen}.
Programming the web with high-level programming languages. 
In {\it Proceedings 2001 European Symposium on Programming\/}, 122--136.
% ------------------------------------------------------------------- 
\litem{2001}  {Clements, J., M. Flatt, and M. Felleisen}.
Modeling an Algebraic Stepper. 
In {\it Proceedings 2001 European Symposium on Programming\/}, 320.-334.

% ------------------------------------------------------------------- 
\litem{2000}  {Krishnamurthi S., M. Felleisen, and B. Duba}.
From macros to reusable generative programming.
In {\it Proceedings First International Symposium on Generative and
Component-Based Software Engineering\/}. Springer Lecture Notes. 
% ------------------------------------------------------------------- 
\litem{1999}  {Flatt, M., R. Findler, S. Krishnamurthi, and M. Felleisen}.
Programming languages as operating systems (or, revenge of the son of the
Lisp machine). 
In {\it Proceedings 1999 International Conference on Functional Programming\/},
138--146. 
% ------------------------------------------------------------------- 
\litem{1999}  {Krishnamurthi S., Y.D. Erlich, and M. Felleisen}.
Expressing structural properties as language constructs.
In {\it Proceedings 1999 European Symposium on Programming\/}, 258--272.

%% in 1998: 
% ------------------------------------------------------------------- 
\litem{1998}  {Krishnamurthi S., and M. Felleisen}.
Toward a formal theory of extensible software.
In {\it Proceedings 1998 Sigsoft Conference on the Foundations of Software
Engineering.\/}  1998, 88--97.
% ------------------------------------------------------------------- 
\litem{1998}  {Krishnamurthi S, M. Felleisen, and D.P. Friedman}.
Synthesizing object-oriented and functional design to promote re-use. 
In {\it Proceedings 1998 European Conference on Object-Oriented Programming.\/}  
Springer Lecture Notes in Computer Science. Berlin, 1998, 91--113.
% ------------------------------------------------------------------- 
\litem{1998}  {Flatt, M., and M. Felleisen}.
Units: Cool modules for HOT languages. 
In {\it Proceedings Sigplan 1998 Conference on Programming Language Design
and Implementation}, 236--248.
% ------------------------------------------------------------------- 
\litem{1998}  {Flatt, M., S. Krishnamurthi, and M. Felleisen}.
Classes and mixins. 
In {\it Proceedings 25th ACM Symposium on Principles of Programming
Languages\/}, 1998, 171--183.
% ------------------------------------------------------------------- 
\litem{1997}  {Findler, R. C. Flanagan, M. Flatt, S. Krishnamurthi, and
M. Felleisen}.
DrScheme: a pedagogic programming environment for Scheme. 
In {\it Proceedings  1997 Symposium on Programming Languages: Implementations and
Logics}, 369-386.
% ------------------------------------------------------------------- 
\litem{1997}  {Flanagan, C., M. Felleisen.} 
Compositional set-based analysis. 
In {\it Proceedings Sigplan 1997 Conference on Programming Language Design
and Implementation}, 235--249.
% ------------------------------------------------------------------- 
\litem{1996}  {Flanagan, C., M. Flatt, S. Krishnamurthi, S. Weirich, and
M. Felleisen}.
Static debugging or browsing the web of program invariants.
In {\it Proceedings Sigplan 1996 Conference on Programming Language Design
and Implementation}, 1996, 23--32.
% ------------------------------------------------------------------- 
\litem{1995}  {Morrisett, G., M. Felleisen, and R. Harper}.
{Modeling memory management}.
In {\it Proceedings Symposium on Functional Programming and Computer
Architectures\/}, 1995, 66--77.  
Also appeared as: Department of Computer Science, Carnegie Mellon
University, Technical Report, 1994. 
% ------------------------------------------------------------------- 
\litem{1995}  {Flanagan, C. and M. Felleisen}.
The semantics of futures and its use in program optimization.
In {\it Proceedings 22nd ACM Symposium on Principles of Programming
Languages\/}, 1995, 209--220.
% ------------------------------------------------------------------- 
\litem{1995}  {Ariola, Z., M. Felleisen, M. Odersky, and P. Wadler}.
The call-by-need $\lambda$-calculus.
In {\it Proceedings 22nd ACM Symposium on Principles of Programming
Languages\/}, 1995, 233--246.
% ------------------------------------------------------------------- 
\litem{1994}  {Sabry, A. and M. Felleisen}.
Is continuation-passing useful for data flow analysis?
In {\it Proceedings SIGPLAN 1993 Conference on Programming Language Design and
Implementation}, 1994, 1--12.
% ------------------------------------------------------------------- 
\litem{1994}  {Cartwright, R., and M. Felleisen}.
Extensible denotational language specifications. 
In {\it Theoretical Aspects of Computer Software\/}.
Lecture Notes in Computer Science~{\bf789}. Springer Verlag, 1994, 244--272.
(Invited Paper)
% ------------------------------------------------------------------- 
\litem{1993} {Flanagan, C., A. Sabry, B.F. Duba, and M. Felleisen}.
The essence of compiling with continuations.
In {\it Proceedings SIGPLAN 1993 Conference on Programming Language Design and
Implementation}, 1993, 237--247.
% ------------------------------------------------------------------- 
\litem{1993} {Weeks, S. and M. Felleisen}.
On the orthogonality of procedures and assignment in Algol~60. 
In {\it Proceedings 20th ACM Symposium on Principles of Programming
Languages\/}, 1993, 57--70.
% ------------------------------------------------------------------- 
\litem{1992} {Kanneganti,R., R.~Cartwright, M.~Felleisen}. 
SPCF: Its Model, Calculus, and Computational Power.
In {\it Proceedings REX Workshop on Semantics and Concurrency\/}. 
Lecture Notes in Computer Science~{\bf666}. Springer Verlag, 1992, 318--347.
(Invited Paper)
% ------------------------------------------------------------------- 
\litem{1992} {Sabry, A. and M. Felleisen}.
Reasoning with programs in continuation-passing style.
In {\it Proceedings 1992 ACM Symposium on Lisp and Functional
Programming\/}, 1992, 288--298.
% ------------------------------------------------------------------- 
\litem{1992} {Cartwright, R. and M. Felleisen}.
Observable sequentiality and full abstraction. 
In {\it Proceedings 19th ACM Symposium on Principles of Programming
Languages\/}, 1992, 328--342.
% ------------------------------------------------------------------- 
\litem{1991}  {Crank, E. and M. Felleisen}.
Parameter-passing techniques and the $\lambda$-calculus. 
In {\it Proceedings 18th ACM Symposium on Principles of Programming
Languages\/}, 1991, 233--245.
% ------------------------------------------------------------------- 
\litem{1991}  {Sitaram, D. and M. Felleisen}.
Modeling continuations without continuations. 
In {\it Proceedings 18th ACM Symposium on Principles of Programming 
Languages\/}, 1991, 185--196.
% ------------------------------------------------------------------- 
\litem{1990}  {Felleisen, M}.
On the expressive power of programming languages.
In {\it Proceedings 1990 European Symposium on Programming}, 
Neil Jones, Ed. Lecture Notes in Computer Science {\bf 432}. Springer
Verlag, 1990, 134--151.  
% ------------------------------------------------------------------- 
\litem{1990}  {Sitaram, D. and M. Felleisen}.
Reasoning with continuations II: Full abstraction for models of
control. 
In {\it Proceedings 1990 Conference on Lisp and Functional Programming}, 
161--175.
% ------------------------------------------------------------------- 
\litem{1989}  {Cartwright, R.S. and M. Felleisen}.
The semantics of program dependence. 
In {\it Proceedings 1989 ACM Conference on the Design and Implementation of
Programming Languages}, 13--27.
% ------------------------------------------------------------------- 
\litem{1988}  {Felleisen, M.} 
$\lambda$-v-CS: An extended $\lambda$-calculus for Scheme.
In {\it Proceedings 1988 Conference on Lisp and Functional Programming}, 
72--85.
% ------------------------------------------------------------------- 
\litem{1988}  {Felleisen, M., M. Wand, D.P. Friedman, and B. Duba.} 
Abstract continuations: A mathematical semantics for handling
full functional jumps.
In {\it Proceedings 1988 Conference on Lisp and Functional Programming}, 
52--62.
% ------------------------------------------------------------------- 
\litem{1988}  {Felleisen, M.}  The theory and practice of first-class prompts.
In {\it Proceedings 15{\it th\/}  Symposium on Principles of Programming
Languages}, 180--190.
% -------------------------------------------------------------------
\litem{1987}  {Felleisen, M. and D.P. Friedman}. 
A reduction semantics for imperative higher-order languages.
In {\it Proceedings Conference on Parallel Architectures and Languages
Europe, Volume II: Parallel Languages}. 
Lecture Notes in Computer Science {\bf 259}.  Springer-Verlag, 
1987, 206--223.
% -------------------------------------------------------------------
\litem{1987} Felleisen, M. and D.P. Friedman. 
A calculus for assignments in higher-order languages.
In {\it Proceedings 14{\it th\/}  Symposium on Principles of Programming
Languages},  314--325.
% -------------------------------------------------------------------
\litem{1986} Felleisen, M., D.P. Friedman, E. Kohlbecker, and B. Duba.
Reasoning with continuations.
In {\it Proceedings First Symposium on Logic in Computer Science}, 131--141.
% -------------------------------------------------------------------
\litem{1986} Kohlbecker, E., D.P. Friedman, M. Felleisen, and  B. Duba.
Hygienic macro expansion.
In {\it Proceedings 1986 Conference on LISP and Functional Programming\/},
151--161. 
% -------------------------------------------------------------------
\endtopic

\topic{Miscellaneous} 

\litem{2015} E. Schanzer, K. Fisler, S. Krishnamurthi, M. Felleisen.
 Transferring Skills at Solving Word Problems from Computing to Algebra
 Through Bootstrap. 
 In {\it Proc. SIGCSE\/}, 2015, 616-621.

\litem{2009} {Eastlund, C., and M. Felleisen}. 
 Automatically Verified GUI Programs. 
 {\it Proceedings Symp. ACL2 Theorem Prover and its Applications}, 2009. 

\litem{2008} {Page, R., Eastlund, C., and M. Felleisen}. 
 Functional programming and theorem proving for undergraduates: a progress report.
 {\it Proceedings Works 2008, Functional and declarative programming in
 education}. 

\litem{2007} {Eastlund, C., Vaillancourt, D., and M. Felleisen}. 
 ACL2 for Freshmen: First Experiences.
 {\it Proceedings Symp. ACL2 Theorem Prover and its Applications}, 2007.

\litem{2006} {Vaillancourt, D., R. Page, M. Felleisen}.
 Dracula: ACL2 in DrScheme. 
 {\it Proceedings Symp. ACL2 Theorem Prover and its Applications}, 2006.

\litem{2006} {Culpepper, R. and M. Felleisen}.
 A Macro Stepper for Sceme 
 {\it Proceedings Worksh. Scheme and Functional Programming}, Portland (OR), 2006. 

\litem{2004} {Flanagan, C., A. Sabry, B.F. Duba, and M. Felleisen}.
The essence of compiling with continuations.
In {\it 20 Years of the ACM SIGPLAN Conference on Programming Language Design
and Implementation (1979 - 1999): A Selection}.  K.S.\ McKinley, Editor,
ACM SIGPLAN Notices,  Volume 39, Number 4, April 2004.

\litem{2004} {Clements, Felleisen, Findler, Flatt, Krishnamurthi.} Fostering little
languages with macros. Dr Dobb's Journal, March 2004. Invited paper. 

\litem{2004} {Felleisen, Findler, Flatt, Krishnamurthi}. Building little
languages with macros. Dr Dobb's Journal, March 2004. Invited paper. 

\litem{2003} {Felleisen, M.} On {\it Gordon Plotkin's Call-By-Name,
 Call-By-Value, and the Lambda Calculus\/}. In {\it Reminiscences on influential
 papers\/}, F.\ Reig and M. \ Franz (eds.). SIGPLAN Notices 2003.

\litem{2002} {M. Felleisen, R. Findler, M. Flatt, and S. Krishnamurthi}.
The structure and interpretation of the computer science curriculum.  In
{\it Proc Workshop Functional and Declarative Programming in
Education}, 2003.

\litem{2001} {Clements, J., P. Graunke, S. Krishnamurthi, and M. Felleisen}. 
Little languages and their programming environments. 
In {\it Proceedings Monterey Workshop}, 2001.

\litem{2001} {Findler, R.B., M. Latendresse, and M. Felleisen}
Object-oriented Programming Languages Need Well-founded Contracts
Rice University, Technical Report 01-372. 

% to report in 99: 

\litem{1999}  {Felleisen, M. and R. Cartwright.}   Safety as a software
metric.  In {\it Proceedings of 12th Conference on Software Engineering
Education and Training \rm(CSEE\&T'99)}, (Invited Position Paper). New
Orleans, March 1999.

\litem{1999}  {Krishmanurthi, S. and M. Felleisen} 
The technology that computer science education overlooked.
In {\it Proceedings of International Conference on Mathematics/Science,
Education and Technology.}  San Antonio, March 1998.

\litem{1998}  {Flanagan, C. and M. Felleisen.} 
A new way of debugging Lisp programs.
In {\it Lisp in the Mainstream: The 40th Annniversary Conference of Lisp
Users\/}. Berkeley, November 1998.

% ------------------------------------------------------------------- 

% reported in 98: 

\litem{1997}  {Felleisen, M. Findler, R. M. Flatt, and Shriram K}.
The DrScheme Project: An Overview.
In {\it SIGPLAN Notices}, June 1998. (Invited Column)
% ------------------------------------------------------------------- 

\litem{1995} Cormac Flanagan, Matthias Felleisen.
Set Based Analysis for Full Scheme and Its Use in Soft-Typing
Rice University, Technical Report TR95-253.

\litem{1995} Cormac Flanagan, Matthias Felleisen.
The Semantics of Futures. 
Rice University, Technical Report TR95-238.

\litem{1995} Cormac Flanagan, Matthias Felleisen.
Well-Founded Touch Optimization of Futures.
Rice University, Technical Report TR95-239.

% \litem{1991} Wright, A. and M. Felleisen.
% A syntactic approach to type soundness. 
% Department of Computer Science, Rice University, Technical Report No
% 160. (submitted to and accepted at Information and Computation)
% \litem{1990} Felleisen, M. and R. Cartwright.  
% Extended direct semantics. 
% Department of Computer Science, Rice University, Technical Report No 105.
% -------------------------------------------------------------------
% \litem{1989} Felleisen, M. and R. Hieb. 
% The revised report on the syntactic theories for control and state.
% Department of Computer Science, Rice University, Technical Report No 100.
% -------------------------------------------------------------------
\litem{1987}  {Felleisen, M.} 
The calculi of $\lambda_v$-CS conversion:
a syntactic theory of control and state in imperative higher-order
programming languages. Computer Science Department, Indiana
University, Technical Report No 226.
% -------------------------------------------------------------------
\litem{1987}  {Duba, B. F., M. Felleisen, and D.P. Friedman.} 
Dynamic identifiers can be neat.
Computer Science Department, Indiana University, Technical Report No 220.
% -------------------------------------------------------------------
\litem{1987}  {Felleisen, M., D.P. Friedman, B. Duba, and J. Merrill},
Beyond continuations.
Computer Science Department, Indiana University, Technical Report No 216.
% -------------------------------------------------------------------
\litem{1986} Felleisen, M.
Recursion and Circularity: Extended Puzzle with Solution.
Computer Science Department, Indiana University, Technical Report 201.
% -------------------------------------------------------------------
\litem{1985} Felleisen, M.
Transliterating Prolog into Scheme.
Computer Science Department, Indiana University, Technical Report 183.
% -------------------------------------------------------------------
% \litem{1983} %
% Felleisen, M.
% Algebraische Spezifikationen als Erg\"anzung 
% des Software-Ent\-wick\-lungs\-kon\-zep\-tes in SARS
% (in German). Master thesis, Universit\"at Karlsruhe.\par
% -------------------------------------------------------------------
% \litem{1982} Felleisen, M.
% Testverfahren im Rahmen von SARS, (in German: Software Testing in SARS).
% Master report, Universit\"at Karlsruhe.\par
% -------------------------------------------------------------------
\endtopic

\topic{Ph.D. Students} 
% 17
\litem{2014} Stephen Chang. {\it On the Relationship Between Laziness and Strictness.\/}\\
% 16
\litem{2012} Christos Dimoulas. {\it Foundations for Behavioral Higher-Order Contracts.\/}\\
% 15
\litem{2012} Carl Eastlund. {\it Modular Proof Development in ACL2.\/}\\
% 14
\litem{2012} T. Stephen Strickland. {\it Scaling Contracts to Realistic Languages.\/}\\
% 13
\litem{2010} Ryan Culpepper. {\it Refining Syntactic Sugar.\/}\\
% 12
\litem{2009} Sam Tobin-Hochstadt. {\it Typed Scheme: From Scripts to Programs.\/}\\
% 11
\litem{2008} Richard Cobbe. {\it Much Ado about Nothing: Putting Null in its Place.\/}\\
% 10
\litem{2006} Philippe Meunier. {\it Modular Set-Based Analysis from Contracts.\/}\\
% 9
\litem{2005} John Clements. {\it Portable and High-level Access to the Stack with Continuation Marks.\/}\\
% 9 
\litem{2003} Paul Graunke. {\it Web Interactions.\/}\\
% 8
\litem{2001} Robert Findler. {\it Behavioral Software Contracts.\/}\\
% 7
\litem{2000} Shriram Krishnamurthi. {\it Language Technology Reuse.\/}\\ 
% 6 
\litem{1999} Matthew Flatt. {\it Programming Languages for Reusable Software Components.\/}\\
% 5
\litem{1997} Cormac Flanagan. {\it Compositional Set-based Analysis for Static Debugging.\/} \\
% 4
\litem{1994} Amr Sabry. {\it The Formal Relationship between Direct and Continuation-passing Style Optimizing Compilers.\/} \\
% 3
\litem{1994} Andrew Wright. {\it Practical Soft Typing\/}. (Joint supervision with Robert Cartwright)\\
% 2
\litem{1994} Dorai Sitaram. {\it Models of Control and Their Implications for Programming Language Design}. \\
% 1
\litem{1992} Rebecca Parsons. {\it A Semantic Framework for Generalized
Program Dependence}. (Joint supervision with Robert Cartwright)\\

%% \newpage

\subtopic{M.S. Dissertations} 

\litem{2014} Claire Alvis. {\it Getting Rid of Undefined}. 
\litem{2010} Hari Prashanth K R. {\it Functional Data Structures in Typed Scheme}. 
\litem{2011} Yue Zoe Zhang. {\it An Attempted Proof with Modular ACL2: Soundness of the Racket Bytecode Verifier}. 
\litem{1990} Erik Crank. {\it Parameter-Passing Techniques and the $\lambda$-Calculus}. 
\litem{1989} Laura Arbilla. {\it A Correspondence between Scheme and the $\lambda_v$-CS-Calculus}. 
\litem{1987} John Gateley. {\it A Call-by-value Combinator Calculus}. (Indiana)

\subtopic{Reader and External Examiner} 
\litem{2003} Danny Dub\'e. {\it Demand-Driven Type Analysis for
Dynamically-Typed Functional Language.\/} Universit\'e de Montr\'eal.
\litem{2000} Arne Kutzner. {\it Ein nichtdeterministischer ``call-by-need''
Kalk{\"ul}  mit ``erratic choice'' Operatoren\/}. Universit{\"a}t Frankfurt.
\litem{1995} Ramarao Kanneganti. {\it A Universal Domain for Sequential
Computation\/}. Rice University.
\litem{1994} Luc Moreau. {\it Sound Evaluation of Parallel
Functional Programs with First-class Continuations\/}. Universit\'e de
Li\`ege. 
\litem{1992} Xavier Leroy. {\it Typage polymorphe d'un langage
algorithmique\/}. Universit\'e Paris~7. 
\litem{1990} Mike Fagan. {\it Soft Typing: An Approach to Type Checking
for Dynamically Typed Languages}. Rice University. 
\endtopic

%% -----------------------------------------------------------------------------

\topic{Invited Conference \& Workshop Keynote Lectures} 

\litem{2016} The Racket Manifesto. 
 CurryOn, Rome (IT), August 2016. 

\litem{2016} Love, Marriage, and Happiness. 
 PLMW @ PLDI, St. Barbara, June 2016.
 PLMW @ OOPSLA, Amsterdam (NL), October 2016. 

\litem{2016} Developing Developers. 
 Trends in Functional Programming/Education, College Park, MD., June 2016.

\litem{2016} Types are Like the Weather,  Type Systems are Like Weathermen. 
 Clojure/West, Seattle, WA, April 2016. 

\litem{2015} Big Bang. Strange Loop, St. Louis, MO, September 2015. 

\litem{2011} Functional Programming is Easy, and Good for You. NYC,
 NY. Goldman Sachs TV, November 2011.

\litem{2011} Multilingual Component Programming in Racket. 
 At the 2011 SIGPLAN Symposium on Generative Programming and Component
 Engineering. Portland, OR. October 2011. 

\litem{2011} The TeachScheme! Project. 
 At the 2011 SIGCSE Conference, Dallas, TX., March 2011. 

\litem{2010} The TeachScheme! Project. 
 At the 2010 International Conference on Functional Programming, Baltimore,
 MD. September 2010. 

\litem{2009} From Soft Scheme to Typed Scheme: Experience from 20 Years of
Script Evolution, and Some Ideas of What Works. 
At the {\it 2009 Scripts to Programs\/}, Genoa, Italy, July 2009. 

\litem{2005} How to Design Class Hierarchies. 
At the {\it 2005 Functional and Declarative Programming in Education\/},
Talinn, Estonia, September 2005. (Delivered by M. Flatt)

\litem{2005} The First Year. At: {\it Proceedings CCSNE\/}, Providence, RI,
June 2005.  

\litem{2004} Functional Classes, Functional Objects. 
At: {\it 2004 European Conference of Object-Oriented Programming\/},
Oslo, Norway, June 2004. 

\litem{2002} Next Generation Software Systems and Programming Languages.
Northeast Programming Languages Workshop. Keynote talk. IBM Watson, May
2002.  

\litem{2002} Next Generation Software Systems and Programming Languages.
{\em Simposium Internacional de Sistemas Computacionales\/}, Monterrey Tech., 
Mexico City, Mexico. Keynote Address. April 2002.

\litem{2002} From POPL to the Classroom and Back. At {\em Symposium on
the Principles of Programming Languages\/}, Portland, OR, Feb 2002. 

\litem{2001} ``Why are they still teaching Scheme when Haskell is so much
better?''  At IFIP Working Group 2.3, 2001, Dartmouth. 

\litem{2000} Static Analysis from one Consumer's Perspective. 
At {\em Static Analysis Symposium\/} 2000, Santa Barbara,
California.  

\litem{1999} The Meaning of Contracts. 
At {\em Inaugural Workshop: Preuves, Programmes, et Systemes\/}, Paris, 
France. 

\litem{1997} TeachScheme! -- A New Approach in Introductory Computing.
At {\it Frontiers in Engineering\/}, Pittsburgh, PA. 

\litem{1994} Extensible Denotational Language Specifications.
At {\it Theoretical Aspects of Computer Software\/}, Sendai, Japan. 

\litem{1993} Expressing and Reasoning about State.
At {\it Workshop on State in Programming Languages\/}, Copenhagen, Denmark. 

\litem{1992} Full Abstraction and Observable Sequentiality: 
An Abstract Characterization of Observably Sequential Function Spaces
and the Computational Power of SPCF.
At {\it North American Jumelage\/}. Cornell University, Ithaca, NY. 

\litem{1992} Sequential PCF: Models and Logics. 
At {\it REX Workshop on Semantics---Foundations and Applications\/}. 
Beekbergen, the Netherlands. 

\litem{1992} Full Abstraction and Observable Sequentiality. 
At {\it Categorical Logic in Computer Science\/}. Aarhus, Denmark.
\endtopic

% {\noindent Numerous Colloquium Talks given, including at the University of
% Wisconsin, Universit{\"a}t Karlsruhe, Universit{\"a}t Frankfurt, University
% of California at Berkeley, MCC Austin, Princeton University, University of
% Texas at Austin, IBM Yorktown Heights, Universit{\"a}t des Saarlandes,
% University of Iowa, Oregon Graduate Institute, University of Pennsylvania,
% Carnegie Mellon University, Harvard, MIT, Boston University, Stanford
% University, Ecole Normale Superieure, Ecole Polytechnique, University of
% Houston, Massachusets Institute of Technology, INRIA Paris, ECRC
% (M{\"u}nchen), GMD Karlsruhe, Cornell University, ATT Bell Labs, University
% of Oregon, Universit{\'e} Li{\`e}ge, Yale University.

% \topic{Refereeing Activities} 

% {\it The Computer Journal\/}, {\it Journal of the ACM\/}, {\it Journal of
% Automated Reasoning\/}, {\it Transactions on Programming Languages and
% Systems\/}, {\it Information \& Computation\/}, {\it Journal of Functional
% Programming\/}, {\it Theoretical Computer Science}, {\it Lisp and Symbolic
% Computation}, {\it Journal of Programming Languages}  (Pergamon Press), {\it
% Software Practice \& Experience}, {\it Mathematical Structures in Computer
% Science\/}, NSF, {\it Scott, Foresman/Little, Brown}, {\it Addison
% Wesley\/}, {\it MIT Press} 
% } 

\end{document} 

\topic{University and Departmental committees} 
\litem{1997--pr.}  Colloquia Committee (Chair: 97-pr.)
\litem{1999--pr.}  Graduate Committee (Chair: 99--pr.)
\litem{1995--1998}  Rice Undergraduate Curriculum Committee
\litem{1994--1995}  Rice Outreach Committee
\litem{1994--1996}  Graduate Policy Committee
\litem{1994--1997}  Faculty Recruiting Committee
\litem{1989--1990}  University Computer Committee
\litem{1989--1993}  Graduate Committee (Chair: 88--90)
\litem{1987--1993}  Colloquia Committee (Chair: 87-93)

\topic{Suggested References} 

\litem{} Pierre-Louis Curien, Ecole Normale Superieure \email{curien@dmi.ens.fr} 
\litem{} Robert Harper, Carnegie Mellon University \email{rwh@cs.cmu.edu} 
\litem{} Paul Hudak, Yale University \email{hudak-paul@cs.yale.edu} 
\litem{} Guy Steele, Sun Microsystems \email{Guy.Steele@east.sun.com} 
\litem{} Mitch Wand, Northeastern University \email{wand@ccs.neu.edu} 

\bigskip

\noindent If an internal reference is necessary, please ask\\
\litem{} Robert ``Corky'' Cartwright, Rice University \email{cork@cs.rice.edu} 

\bigskip

\noindent If you wish to request references from some of my former Ph.D.
students, please ask\\

\litem{} Amr Sabry, University of Oregon \email{sabry@cs.uoregon.edu} 
\litem{} Cormac Flanagan, DEC SRC \email{flanagan@pa.dec.com} 
\litem{} Andrew Wright, Intertrust \email{wright@intertrust.com} 
