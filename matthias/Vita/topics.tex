
\specialtopic{The TeachScheme! Project} 

In January 1996, I initiated the TeachScheme!\ project. The goal of the
project is to train high school teachers and college professors in modern,
data-centric program design methods using Scheme (and some Java). The basic
design method for functions consists of six steps: data analysis and
definition; function contract and specification (types plus a one line
comment); examples; function layout; function body; and testing. The most
innovative aspect of the design method is a systematic explanation of the
function's layout in terms of the data definitions; this step itself is
also formulated as a (three-step) recipe. The design method works equally
well for functional program design and object-oriented program design,
especially in the context of the model-view-controller pattern.

The design method is based on theory and practice. More precisely,
denotational and operational semantics, plus type theory, provide the
theoretical foundations. The recent work on design patterns supplies the
practical justification, because the use of patterns produces nearly
identical program organizations.  Observations of beginning students (at
all ability levels) have suggested the formulations of the design
recipes. The new design method is subject of the book ``How to Design
Programs'', MIT Press 2000. 

Since 1997 I have organized and taught six workshops; the enrollment has
grown from nine to sixty per year, mostly due to word-of-mouth
advertising. My team and I have trained almost 100 teachers. They report
tremendous success with the design method. For example, using the design
recipes teachers have covered linked-list and tree processing with eighth
graders and high school students in a single semester; these topics are
typically out of reach until the second or third year of high school
computing.  Also, because of its emphasis on program construction and the
de-emphasis of program syntax, the curriculum seems to appeal to a much
wider audience than the traditional AP/C++ curriculum. Especially girls
prefer the new curriculum by a wide margin.

By now, the project has won national and international publicity.
Attendees come from American schools all over the US, Europe, Mexico, and
Japan and from colleges around the nation. I have been invited to give
numerous talks on the projects, which are always well received.  Most
recently O.\ Astrachan (Duke) organized a conference on first-year
curricula and invited me to present one of three key-note lectures. 

With the help of NSF I have created a nationwide dissemination effort.
Faculty at Adelphi University, Brown University, the University of
Utah, and WPI have establishd satellite centers; others are currently
being recruited.  

\endtopic

\specialtopic{The DrScheme Project} 

In support of the TeachScheme! effort, I also started a new research
effort: the development of DrScheme, a fully portable, graphical
programming environment. The effort resulted from the observation that all
programming courses teach languages in a layered fashion, that no
programming environment supports this layering with enforcement mechanisms,
and that all students, including the best, suffer from simple syntactic
slips because they result in meaningless results or in meaningless error
messages about portions of the language they don't know. 

DrScheme addresses this problem with a tower of increasingly complex
programming languages, each self-contained, so that all error reporting
occurs at a learner-appropriate level. The introduction of DrScheme in
Rice's courses tremendously improved the quality of instruction.  DrScheme
is now used by around 140 universities, including all (but one) of those
universities that host their own Scheme implementation; at a minimum of 80
high schools; and at around 20 industrial sites (as a retraining tool). [My
team only counts those sites that sent an acknowledgement.] The fact that
DrScheme is used by around 10,000 students on a daily basis has provided
significant feedback and has directed my team's attention to essential
issues.

In the meantime, DrScheme has grown from its modest beginnings as a
pedagogic programming environment into an umbrella project that provides us
with a wealth of research opportunities on software engineering and
programming languages.  Over the past five years, we have used the
environment to conduct research on: 
language constructs for extensible programming; 
program organizations for extensible products; 
language support for component programming; 
system support for dynamically growing component systems; 
static analyses and debugging; 
optimizing compilers; 
programming environments; 
algebraic stepping.

\endtopic
